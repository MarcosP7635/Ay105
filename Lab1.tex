\documentclass[11pt]{article}

    \usepackage[breakable]{tcolorbox}
    \usepackage{parskip} % Stop auto-indenting (to mimic markdown behaviour)
    

    % Basic figure setup, for now with no caption control since it's done
    % automatically by Pandoc (which extracts ![](path) syntax from Markdown).
    \usepackage{graphicx}
    % Maintain compatibility with old templates. Remove in nbconvert 6.0
    \let\Oldincludegraphics\includegraphics
    % Ensure that by default, figures have no caption (until we provide a
    % proper Figure object with a Caption API and a way to capture that
    % in the conversion process - todo).
    \usepackage{caption}
    \DeclareCaptionFormat{nocaption}{}
    \captionsetup{format=nocaption,aboveskip=0pt,belowskip=0pt}

    \usepackage{float}
    \floatplacement{figure}{H} % forces figures to be placed at the correct location
    \usepackage{xcolor} % Allow colors to be defined
    \usepackage{enumerate} % Needed for markdown enumerations to work
    \usepackage{geometry} % Used to adjust the document margins
    \usepackage{amsmath} % Equations
    \usepackage{amssymb} % Equations
    \usepackage{textcomp} % defines textquotesingle
    % Hack from http://tex.stackexchange.com/a/47451/13684:
    \AtBeginDocument{%
        \def\PYZsq{\textquotesingle}% Upright quotes in Pygmentized code
    }
    \usepackage{upquote} % Upright quotes for verbatim code
    \usepackage{eurosym} % defines \euro

    \usepackage{iftex}
    \ifPDFTeX
        \usepackage[T1]{fontenc}
        \IfFileExists{alphabeta.sty}{
              \usepackage{alphabeta}
          }{
              \usepackage[mathletters]{ucs}
              \usepackage[utf8x]{inputenc}
          }
    \else
        \usepackage{fontspec}
        \usepackage{unicode-math}
    \fi

    \usepackage{fancyvrb} % verbatim replacement that allows latex
    \usepackage{grffile} % extends the file name processing of package graphics 
                         % to support a larger range
    \makeatletter % fix for old versions of grffile with XeLaTeX
    \@ifpackagelater{grffile}{2019/11/01}
    {
      % Do nothing on new versions
    }
    {
      \def\Gread@@xetex#1{%
        \IfFileExists{"\Gin@base".bb}%
        {\Gread@eps{\Gin@base.bb}}%
        {\Gread@@xetex@aux#1}%
      }
    }
    \makeatother
    \usepackage[Export]{adjustbox} % Used to constrain images to a maximum size
    \adjustboxset{max size={0.9\linewidth}{0.9\paperheight}}

    % The hyperref package gives us a pdf with properly built
    % internal navigation ('pdf bookmarks' for the table of contents,
    % internal cross-reference links, web links for URLs, etc.)
    \usepackage{hyperref}
    % The default LaTeX title has an obnoxious amount of whitespace. By default,
    % titling removes some of it. It also provides customization options.
    \usepackage{titling}
    \usepackage{longtable} % longtable support required by pandoc >1.10
    \usepackage{booktabs}  % table support for pandoc > 1.12.2
    \usepackage{array}     % table support for pandoc >= 2.11.3
    \usepackage{calc}      % table minipage width calculation for pandoc >= 2.11.1
    \usepackage[inline]{enumitem} % IRkernel/repr support (it uses the enumerate* environment)
    \usepackage[normalem]{ulem} % ulem is needed to support strikethroughs (\sout)
                                % normalem makes italics be italics, not underlines
    \usepackage{mathrsfs}
    

    
    % Colors for the hyperref package
    \definecolor{urlcolor}{rgb}{0,.145,.698}
    \definecolor{linkcolor}{rgb}{.71,0.21,0.01}
    \definecolor{citecolor}{rgb}{.12,.54,.11}

    % ANSI colors
    \definecolor{ansi-black}{HTML}{3E424D}
    \definecolor{ansi-black-intense}{HTML}{282C36}
    \definecolor{ansi-red}{HTML}{E75C58}
    \definecolor{ansi-red-intense}{HTML}{B22B31}
    \definecolor{ansi-green}{HTML}{00A250}
    \definecolor{ansi-green-intense}{HTML}{007427}
    \definecolor{ansi-yellow}{HTML}{DDB62B}
    \definecolor{ansi-yellow-intense}{HTML}{B27D12}
    \definecolor{ansi-blue}{HTML}{208FFB}
    \definecolor{ansi-blue-intense}{HTML}{0065CA}
    \definecolor{ansi-magenta}{HTML}{D160C4}
    \definecolor{ansi-magenta-intense}{HTML}{A03196}
    \definecolor{ansi-cyan}{HTML}{60C6C8}
    \definecolor{ansi-cyan-intense}{HTML}{258F8F}
    \definecolor{ansi-white}{HTML}{C5C1B4}
    \definecolor{ansi-white-intense}{HTML}{A1A6B2}
    \definecolor{ansi-default-inverse-fg}{HTML}{FFFFFF}
    \definecolor{ansi-default-inverse-bg}{HTML}{000000}

    % common color for the border for error outputs.
    \definecolor{outerrorbackground}{HTML}{FFDFDF}

    % commands and environments needed by pandoc snippets
    % extracted from the output of `pandoc -s`
    \providecommand{\tightlist}{%
      \setlength{\itemsep}{0pt}\setlength{\parskip}{0pt}}
    \DefineVerbatimEnvironment{Highlighting}{Verbatim}{commandchars=\\\{\}}
    % Add ',fontsize=\small' for more characters per line
    \newenvironment{Shaded}{}{}
    \newcommand{\KeywordTok}[1]{\textcolor[rgb]{0.00,0.44,0.13}{\textbf{{#1}}}}
    \newcommand{\DataTypeTok}[1]{\textcolor[rgb]{0.56,0.13,0.00}{{#1}}}
    \newcommand{\DecValTok}[1]{\textcolor[rgb]{0.25,0.63,0.44}{{#1}}}
    \newcommand{\BaseNTok}[1]{\textcolor[rgb]{0.25,0.63,0.44}{{#1}}}
    \newcommand{\FloatTok}[1]{\textcolor[rgb]{0.25,0.63,0.44}{{#1}}}
    \newcommand{\CharTok}[1]{\textcolor[rgb]{0.25,0.44,0.63}{{#1}}}
    \newcommand{\StringTok}[1]{\textcolor[rgb]{0.25,0.44,0.63}{{#1}}}
    \newcommand{\CommentTok}[1]{\textcolor[rgb]{0.38,0.63,0.69}{\textit{{#1}}}}
    \newcommand{\OtherTok}[1]{\textcolor[rgb]{0.00,0.44,0.13}{{#1}}}
    \newcommand{\AlertTok}[1]{\textcolor[rgb]{1.00,0.00,0.00}{\textbf{{#1}}}}
    \newcommand{\FunctionTok}[1]{\textcolor[rgb]{0.02,0.16,0.49}{{#1}}}
    \newcommand{\RegionMarkerTok}[1]{{#1}}
    \newcommand{\ErrorTok}[1]{\textcolor[rgb]{1.00,0.00,0.00}{\textbf{{#1}}}}
    \newcommand{\NormalTok}[1]{{#1}}
    
    % Additional commands for more recent versions of Pandoc
    \newcommand{\ConstantTok}[1]{\textcolor[rgb]{0.53,0.00,0.00}{{#1}}}
    \newcommand{\SpecialCharTok}[1]{\textcolor[rgb]{0.25,0.44,0.63}{{#1}}}
    \newcommand{\VerbatimStringTok}[1]{\textcolor[rgb]{0.25,0.44,0.63}{{#1}}}
    \newcommand{\SpecialStringTok}[1]{\textcolor[rgb]{0.73,0.40,0.53}{{#1}}}
    \newcommand{\ImportTok}[1]{{#1}}
    \newcommand{\DocumentationTok}[1]{\textcolor[rgb]{0.73,0.13,0.13}{\textit{{#1}}}}
    \newcommand{\AnnotationTok}[1]{\textcolor[rgb]{0.38,0.63,0.69}{\textbf{\textit{{#1}}}}}
    \newcommand{\CommentVarTok}[1]{\textcolor[rgb]{0.38,0.63,0.69}{\textbf{\textit{{#1}}}}}
    \newcommand{\VariableTok}[1]{\textcolor[rgb]{0.10,0.09,0.49}{{#1}}}
    \newcommand{\ControlFlowTok}[1]{\textcolor[rgb]{0.00,0.44,0.13}{\textbf{{#1}}}}
    \newcommand{\OperatorTok}[1]{\textcolor[rgb]{0.40,0.40,0.40}{{#1}}}
    \newcommand{\BuiltInTok}[1]{{#1}}
    \newcommand{\ExtensionTok}[1]{{#1}}
    \newcommand{\PreprocessorTok}[1]{\textcolor[rgb]{0.74,0.48,0.00}{{#1}}}
    \newcommand{\AttributeTok}[1]{\textcolor[rgb]{0.49,0.56,0.16}{{#1}}}
    \newcommand{\InformationTok}[1]{\textcolor[rgb]{0.38,0.63,0.69}{\textbf{\textit{{#1}}}}}
    \newcommand{\WarningTok}[1]{\textcolor[rgb]{0.38,0.63,0.69}{\textbf{\textit{{#1}}}}}
    
    
    % Define a nice break command that doesn't care if a line doesn't already
    % exist.
    \def\br{\hspace*{\fill} \\* }
    % Math Jax compatibility definitions
    \def\gt{>}
    \def\lt{<}
    \let\Oldtex\TeX
    \let\Oldlatex\LaTeX
    \renewcommand{\TeX}{\textrm{\Oldtex}}
    \renewcommand{\LaTeX}{\textrm{\Oldlatex}}
    % Document parameters
    % Document title
    \title{Lab1}
    
    
    
    
    
% Pygments definitions
\makeatletter
\def\PY@reset{\let\PY@it=\relax \let\PY@bf=\relax%
    \let\PY@ul=\relax \let\PY@tc=\relax%
    \let\PY@bc=\relax \let\PY@ff=\relax}
\def\PY@tok#1{\csname PY@tok@#1\endcsname}
\def\PY@toks#1+{\ifx\relax#1\empty\else%
    \PY@tok{#1}\expandafter\PY@toks\fi}
\def\PY@do#1{\PY@bc{\PY@tc{\PY@ul{%
    \PY@it{\PY@bf{\PY@ff{#1}}}}}}}
\def\PY#1#2{\PY@reset\PY@toks#1+\relax+\PY@do{#2}}

\@namedef{PY@tok@w}{\def\PY@tc##1{\textcolor[rgb]{0.73,0.73,0.73}{##1}}}
\@namedef{PY@tok@c}{\let\PY@it=\textit\def\PY@tc##1{\textcolor[rgb]{0.24,0.48,0.48}{##1}}}
\@namedef{PY@tok@cp}{\def\PY@tc##1{\textcolor[rgb]{0.61,0.40,0.00}{##1}}}
\@namedef{PY@tok@k}{\let\PY@bf=\textbf\def\PY@tc##1{\textcolor[rgb]{0.00,0.50,0.00}{##1}}}
\@namedef{PY@tok@kp}{\def\PY@tc##1{\textcolor[rgb]{0.00,0.50,0.00}{##1}}}
\@namedef{PY@tok@kt}{\def\PY@tc##1{\textcolor[rgb]{0.69,0.00,0.25}{##1}}}
\@namedef{PY@tok@o}{\def\PY@tc##1{\textcolor[rgb]{0.40,0.40,0.40}{##1}}}
\@namedef{PY@tok@ow}{\let\PY@bf=\textbf\def\PY@tc##1{\textcolor[rgb]{0.67,0.13,1.00}{##1}}}
\@namedef{PY@tok@nb}{\def\PY@tc##1{\textcolor[rgb]{0.00,0.50,0.00}{##1}}}
\@namedef{PY@tok@nf}{\def\PY@tc##1{\textcolor[rgb]{0.00,0.00,1.00}{##1}}}
\@namedef{PY@tok@nc}{\let\PY@bf=\textbf\def\PY@tc##1{\textcolor[rgb]{0.00,0.00,1.00}{##1}}}
\@namedef{PY@tok@nn}{\let\PY@bf=\textbf\def\PY@tc##1{\textcolor[rgb]{0.00,0.00,1.00}{##1}}}
\@namedef{PY@tok@ne}{\let\PY@bf=\textbf\def\PY@tc##1{\textcolor[rgb]{0.80,0.25,0.22}{##1}}}
\@namedef{PY@tok@nv}{\def\PY@tc##1{\textcolor[rgb]{0.10,0.09,0.49}{##1}}}
\@namedef{PY@tok@no}{\def\PY@tc##1{\textcolor[rgb]{0.53,0.00,0.00}{##1}}}
\@namedef{PY@tok@nl}{\def\PY@tc##1{\textcolor[rgb]{0.46,0.46,0.00}{##1}}}
\@namedef{PY@tok@ni}{\let\PY@bf=\textbf\def\PY@tc##1{\textcolor[rgb]{0.44,0.44,0.44}{##1}}}
\@namedef{PY@tok@na}{\def\PY@tc##1{\textcolor[rgb]{0.41,0.47,0.13}{##1}}}
\@namedef{PY@tok@nt}{\let\PY@bf=\textbf\def\PY@tc##1{\textcolor[rgb]{0.00,0.50,0.00}{##1}}}
\@namedef{PY@tok@nd}{\def\PY@tc##1{\textcolor[rgb]{0.67,0.13,1.00}{##1}}}
\@namedef{PY@tok@s}{\def\PY@tc##1{\textcolor[rgb]{0.73,0.13,0.13}{##1}}}
\@namedef{PY@tok@sd}{\let\PY@it=\textit\def\PY@tc##1{\textcolor[rgb]{0.73,0.13,0.13}{##1}}}
\@namedef{PY@tok@si}{\let\PY@bf=\textbf\def\PY@tc##1{\textcolor[rgb]{0.64,0.35,0.47}{##1}}}
\@namedef{PY@tok@se}{\let\PY@bf=\textbf\def\PY@tc##1{\textcolor[rgb]{0.67,0.36,0.12}{##1}}}
\@namedef{PY@tok@sr}{\def\PY@tc##1{\textcolor[rgb]{0.64,0.35,0.47}{##1}}}
\@namedef{PY@tok@ss}{\def\PY@tc##1{\textcolor[rgb]{0.10,0.09,0.49}{##1}}}
\@namedef{PY@tok@sx}{\def\PY@tc##1{\textcolor[rgb]{0.00,0.50,0.00}{##1}}}
\@namedef{PY@tok@m}{\def\PY@tc##1{\textcolor[rgb]{0.40,0.40,0.40}{##1}}}
\@namedef{PY@tok@gh}{\let\PY@bf=\textbf\def\PY@tc##1{\textcolor[rgb]{0.00,0.00,0.50}{##1}}}
\@namedef{PY@tok@gu}{\let\PY@bf=\textbf\def\PY@tc##1{\textcolor[rgb]{0.50,0.00,0.50}{##1}}}
\@namedef{PY@tok@gd}{\def\PY@tc##1{\textcolor[rgb]{0.63,0.00,0.00}{##1}}}
\@namedef{PY@tok@gi}{\def\PY@tc##1{\textcolor[rgb]{0.00,0.52,0.00}{##1}}}
\@namedef{PY@tok@gr}{\def\PY@tc##1{\textcolor[rgb]{0.89,0.00,0.00}{##1}}}
\@namedef{PY@tok@ge}{\let\PY@it=\textit}
\@namedef{PY@tok@gs}{\let\PY@bf=\textbf}
\@namedef{PY@tok@gp}{\let\PY@bf=\textbf\def\PY@tc##1{\textcolor[rgb]{0.00,0.00,0.50}{##1}}}
\@namedef{PY@tok@go}{\def\PY@tc##1{\textcolor[rgb]{0.44,0.44,0.44}{##1}}}
\@namedef{PY@tok@gt}{\def\PY@tc##1{\textcolor[rgb]{0.00,0.27,0.87}{##1}}}
\@namedef{PY@tok@err}{\def\PY@bc##1{{\setlength{\fboxsep}{\string -\fboxrule}\fcolorbox[rgb]{1.00,0.00,0.00}{1,1,1}{\strut ##1}}}}
\@namedef{PY@tok@kc}{\let\PY@bf=\textbf\def\PY@tc##1{\textcolor[rgb]{0.00,0.50,0.00}{##1}}}
\@namedef{PY@tok@kd}{\let\PY@bf=\textbf\def\PY@tc##1{\textcolor[rgb]{0.00,0.50,0.00}{##1}}}
\@namedef{PY@tok@kn}{\let\PY@bf=\textbf\def\PY@tc##1{\textcolor[rgb]{0.00,0.50,0.00}{##1}}}
\@namedef{PY@tok@kr}{\let\PY@bf=\textbf\def\PY@tc##1{\textcolor[rgb]{0.00,0.50,0.00}{##1}}}
\@namedef{PY@tok@bp}{\def\PY@tc##1{\textcolor[rgb]{0.00,0.50,0.00}{##1}}}
\@namedef{PY@tok@fm}{\def\PY@tc##1{\textcolor[rgb]{0.00,0.00,1.00}{##1}}}
\@namedef{PY@tok@vc}{\def\PY@tc##1{\textcolor[rgb]{0.10,0.09,0.49}{##1}}}
\@namedef{PY@tok@vg}{\def\PY@tc##1{\textcolor[rgb]{0.10,0.09,0.49}{##1}}}
\@namedef{PY@tok@vi}{\def\PY@tc##1{\textcolor[rgb]{0.10,0.09,0.49}{##1}}}
\@namedef{PY@tok@vm}{\def\PY@tc##1{\textcolor[rgb]{0.10,0.09,0.49}{##1}}}
\@namedef{PY@tok@sa}{\def\PY@tc##1{\textcolor[rgb]{0.73,0.13,0.13}{##1}}}
\@namedef{PY@tok@sb}{\def\PY@tc##1{\textcolor[rgb]{0.73,0.13,0.13}{##1}}}
\@namedef{PY@tok@sc}{\def\PY@tc##1{\textcolor[rgb]{0.73,0.13,0.13}{##1}}}
\@namedef{PY@tok@dl}{\def\PY@tc##1{\textcolor[rgb]{0.73,0.13,0.13}{##1}}}
\@namedef{PY@tok@s2}{\def\PY@tc##1{\textcolor[rgb]{0.73,0.13,0.13}{##1}}}
\@namedef{PY@tok@sh}{\def\PY@tc##1{\textcolor[rgb]{0.73,0.13,0.13}{##1}}}
\@namedef{PY@tok@s1}{\def\PY@tc##1{\textcolor[rgb]{0.73,0.13,0.13}{##1}}}
\@namedef{PY@tok@mb}{\def\PY@tc##1{\textcolor[rgb]{0.40,0.40,0.40}{##1}}}
\@namedef{PY@tok@mf}{\def\PY@tc##1{\textcolor[rgb]{0.40,0.40,0.40}{##1}}}
\@namedef{PY@tok@mh}{\def\PY@tc##1{\textcolor[rgb]{0.40,0.40,0.40}{##1}}}
\@namedef{PY@tok@mi}{\def\PY@tc##1{\textcolor[rgb]{0.40,0.40,0.40}{##1}}}
\@namedef{PY@tok@il}{\def\PY@tc##1{\textcolor[rgb]{0.40,0.40,0.40}{##1}}}
\@namedef{PY@tok@mo}{\def\PY@tc##1{\textcolor[rgb]{0.40,0.40,0.40}{##1}}}
\@namedef{PY@tok@ch}{\let\PY@it=\textit\def\PY@tc##1{\textcolor[rgb]{0.24,0.48,0.48}{##1}}}
\@namedef{PY@tok@cm}{\let\PY@it=\textit\def\PY@tc##1{\textcolor[rgb]{0.24,0.48,0.48}{##1}}}
\@namedef{PY@tok@cpf}{\let\PY@it=\textit\def\PY@tc##1{\textcolor[rgb]{0.24,0.48,0.48}{##1}}}
\@namedef{PY@tok@c1}{\let\PY@it=\textit\def\PY@tc##1{\textcolor[rgb]{0.24,0.48,0.48}{##1}}}
\@namedef{PY@tok@cs}{\let\PY@it=\textit\def\PY@tc##1{\textcolor[rgb]{0.24,0.48,0.48}{##1}}}

\def\PYZbs{\char`\\}
\def\PYZus{\char`\_}
\def\PYZob{\char`\{}
\def\PYZcb{\char`\}}
\def\PYZca{\char`\^}
\def\PYZam{\char`\&}
\def\PYZlt{\char`\<}
\def\PYZgt{\char`\>}
\def\PYZsh{\char`\#}
\def\PYZpc{\char`\%}
\def\PYZdl{\char`\$}
\def\PYZhy{\char`\-}
\def\PYZsq{\char`\'}
\def\PYZdq{\char`\"}
\def\PYZti{\char`\~}
% for compatibility with earlier versions
\def\PYZat{@}
\def\PYZlb{[}
\def\PYZrb{]}
\makeatother


    % For linebreaks inside Verbatim environment from package fancyvrb. 
    \makeatletter
        \newbox\Wrappedcontinuationbox 
        \newbox\Wrappedvisiblespacebox 
        \newcommand*\Wrappedvisiblespace {\textcolor{red}{\textvisiblespace}} 
        \newcommand*\Wrappedcontinuationsymbol {\textcolor{red}{\llap{\tiny$\m@th\hookrightarrow$}}} 
        \newcommand*\Wrappedcontinuationindent {3ex } 
        \newcommand*\Wrappedafterbreak {\kern\Wrappedcontinuationindent\copy\Wrappedcontinuationbox} 
        % Take advantage of the already applied Pygments mark-up to insert 
        % potential linebreaks for TeX processing. 
        %        {, <, #, %, $, ' and ": go to next line. 
        %        _, }, ^, &, >, - and ~: stay at end of broken line. 
        % Use of \textquotesingle for straight quote. 
        \newcommand*\Wrappedbreaksatspecials {% 
            \def\PYGZus{\discretionary{\char`\_}{\Wrappedafterbreak}{\char`\_}}% 
            \def\PYGZob{\discretionary{}{\Wrappedafterbreak\char`\{}{\char`\{}}% 
            \def\PYGZcb{\discretionary{\char`\}}{\Wrappedafterbreak}{\char`\}}}% 
            \def\PYGZca{\discretionary{\char`\^}{\Wrappedafterbreak}{\char`\^}}% 
            \def\PYGZam{\discretionary{\char`\&}{\Wrappedafterbreak}{\char`\&}}% 
            \def\PYGZlt{\discretionary{}{\Wrappedafterbreak\char`\<}{\char`\<}}% 
            \def\PYGZgt{\discretionary{\char`\>}{\Wrappedafterbreak}{\char`\>}}% 
            \def\PYGZsh{\discretionary{}{\Wrappedafterbreak\char`\#}{\char`\#}}% 
            \def\PYGZpc{\discretionary{}{\Wrappedafterbreak\char`\%}{\char`\%}}% 
            \def\PYGZdl{\discretionary{}{\Wrappedafterbreak\char`\$}{\char`\$}}% 
            \def\PYGZhy{\discretionary{\char`\-}{\Wrappedafterbreak}{\char`\-}}% 
            \def\PYGZsq{\discretionary{}{\Wrappedafterbreak\textquotesingle}{\textquotesingle}}% 
            \def\PYGZdq{\discretionary{}{\Wrappedafterbreak\char`\"}{\char`\"}}% 
            \def\PYGZti{\discretionary{\char`\~}{\Wrappedafterbreak}{\char`\~}}% 
        } 
        % Some characters . , ; ? ! / are not pygmentized. 
        % This macro makes them "active" and they will insert potential linebreaks 
        \newcommand*\Wrappedbreaksatpunct {% 
            \lccode`\~`\.\lowercase{\def~}{\discretionary{\hbox{\char`\.}}{\Wrappedafterbreak}{\hbox{\char`\.}}}% 
            \lccode`\~`\,\lowercase{\def~}{\discretionary{\hbox{\char`\,}}{\Wrappedafterbreak}{\hbox{\char`\,}}}% 
            \lccode`\~`\;\lowercase{\def~}{\discretionary{\hbox{\char`\;}}{\Wrappedafterbreak}{\hbox{\char`\;}}}% 
            \lccode`\~`\:\lowercase{\def~}{\discretionary{\hbox{\char`\:}}{\Wrappedafterbreak}{\hbox{\char`\:}}}% 
            \lccode`\~`\?\lowercase{\def~}{\discretionary{\hbox{\char`\?}}{\Wrappedafterbreak}{\hbox{\char`\?}}}% 
            \lccode`\~`\!\lowercase{\def~}{\discretionary{\hbox{\char`\!}}{\Wrappedafterbreak}{\hbox{\char`\!}}}% 
            \lccode`\~`\/\lowercase{\def~}{\discretionary{\hbox{\char`\/}}{\Wrappedafterbreak}{\hbox{\char`\/}}}% 
            \catcode`\.\active
            \catcode`\,\active 
            \catcode`\;\active
            \catcode`\:\active
            \catcode`\?\active
            \catcode`\!\active
            \catcode`\/\active 
            \lccode`\~`\~ 	
        }
    \makeatother

    \let\OriginalVerbatim=\Verbatim
    \makeatletter
    \renewcommand{\Verbatim}[1][1]{%
        %\parskip\z@skip
        \sbox\Wrappedcontinuationbox {\Wrappedcontinuationsymbol}%
        \sbox\Wrappedvisiblespacebox {\FV@SetupFont\Wrappedvisiblespace}%
        \def\FancyVerbFormatLine ##1{\hsize\linewidth
            \vtop{\raggedright\hyphenpenalty\z@\exhyphenpenalty\z@
                \doublehyphendemerits\z@\finalhyphendemerits\z@
                \strut ##1\strut}%
        }%
        % If the linebreak is at a space, the latter will be displayed as visible
        % space at end of first line, and a continuation symbol starts next line.
        % Stretch/shrink are however usually zero for typewriter font.
        \def\FV@Space {%
            \nobreak\hskip\z@ plus\fontdimen3\font minus\fontdimen4\font
            \discretionary{\copy\Wrappedvisiblespacebox}{\Wrappedafterbreak}
            {\kern\fontdimen2\font}%
        }%
        
        % Allow breaks at special characters using \PYG... macros.
        \Wrappedbreaksatspecials
        % Breaks at punctuation characters . , ; ? ! and / need catcode=\active 	
        \OriginalVerbatim[#1,codes*=\Wrappedbreaksatpunct]%
    }
    \makeatother

    % Exact colors from NB
    \definecolor{incolor}{HTML}{303F9F}
    \definecolor{outcolor}{HTML}{D84315}
    \definecolor{cellborder}{HTML}{CFCFCF}
    \definecolor{cellbackground}{HTML}{F7F7F7}
    
    % prompt
    \makeatletter
    \newcommand{\boxspacing}{\kern\kvtcb@left@rule\kern\kvtcb@boxsep}
    \makeatother
    \newcommand{\prompt}[4]{
        {\ttfamily\llap{{\color{#2}[#3]:\hspace{3pt}#4}}\vspace{-\baselineskip}}
    }
    

    
    % Prevent overflowing lines due to hard-to-break entities
    \sloppy 
    % Setup hyperref package
    \hypersetup{
      breaklinks=true,  % so long urls are correctly broken across lines
      colorlinks=true,
      urlcolor=urlcolor,
      linkcolor=linkcolor,
      citecolor=citecolor,
      }
    % Slightly bigger margins than the latex defaults
    
    \geometry{verbose,tmargin=1in,bmargin=1in,lmargin=1in,rmargin=1in}
    
    

\begin{document}
    
    \maketitle
    
    

    
    \begin{tcolorbox}[breakable, size=fbox, boxrule=1pt, pad at break*=1mm,colback=cellbackground, colframe=cellborder]
\prompt{In}{incolor}{1}{\boxspacing}
\begin{Verbatim}[commandchars=\\\{\}]
\PY{k+kn}{import} \PY{n+nn}{numpy} \PY{k}{as} \PY{n+nn}{np}\PY{p}{;}
\PY{k+kn}{import} \PY{n+nn}{matplotlib}\PY{n+nn}{.}\PY{n+nn}{pyplot} \PY{k}{as} \PY{n+nn}{plt}
\PY{k+kn}{from} \PY{n+nn}{astropy}\PY{n+nn}{.}\PY{n+nn}{io} \PY{k+kn}{import} \PY{n}{fits}
\PY{k+kn}{import} \PY{n+nn}{math}
\PY{k+kn}{from} \PY{n+nn}{scipy}\PY{n+nn}{.}\PY{n+nn}{optimize} \PY{k+kn}{import} \PY{n}{curve\PYZus{}fit}
\PY{k+kn}{import} \PY{n+nn}{glob}
\PY{k+kn}{import} \PY{n+nn}{os}
\PY{k+kn}{import} \PY{n+nn}{sys}
\PY{k+kn}{import} \PY{n+nn}{matplotlib}\PY{n+nn}{.}\PY{n+nn}{backends}\PY{n+nn}{.}\PY{n+nn}{backend\PYZus{}pdf}
\PY{k+kn}{import} \PY{n+nn}{matplotlib}\PY{n+nn}{.}\PY{n+nn}{gridspec} \PY{k}{as} \PY{n+nn}{gridspec}
\PY{k+kn}{from} \PY{n+nn}{scipy}\PY{n+nn}{.}\PY{n+nn}{stats} \PY{k+kn}{import} \PY{n}{norm}
\PY{k+kn}{from} \PY{n+nn}{scipy}\PY{n+nn}{.}\PY{n+nn}{stats} \PY{k+kn}{import} \PY{n}{lognorm}
\PY{k+kn}{from} \PY{n+nn}{scipy} \PY{k+kn}{import} \PY{n}{linalg}
\PY{k+kn}{import} \PY{n+nn}{pprint}
\PY{k+kn}{import} \PY{n+nn}{re}
\PY{n}{plt}\PY{o}{.}\PY{n}{rc}\PY{p}{(}\PY{l+s+s1}{\PYZsq{}}\PY{l+s+s1}{font}\PY{l+s+s1}{\PYZsq{}}\PY{p}{,} \PY{n}{family}\PY{o}{=}\PY{l+s+s1}{\PYZsq{}}\PY{l+s+s1}{serif}\PY{l+s+s1}{\PYZsq{}}\PY{p}{)}
\PY{k+kn}{from} \PY{n+nn}{matplotlib} \PY{k+kn}{import} \PY{n}{rc}
\PY{k+kn}{import} \PY{n+nn}{pandas} \PY{k}{as} \PY{n+nn}{pd}
\PY{k+kn}{from} \PY{n+nn}{astropy} \PY{k+kn}{import} \PY{n}{constants} \PY{k}{as} \PY{n}{const}
\PY{k+kn}{from} \PY{n+nn}{astropy} \PY{k+kn}{import} \PY{n}{units} \PY{k}{as} \PY{n}{units}
\PY{k+kn}{from} \PY{n+nn}{IPython}\PY{n+nn}{.}\PY{n+nn}{display} \PY{k+kn}{import} \PY{n}{Markdown} \PY{k}{as} \PY{n}{md}
\PY{k+kn}{import} \PY{n+nn}{sympy}
\PY{k+kn}{import} \PY{n+nn}{pandas}
\PY{k+kn}{from} \PY{n+nn}{sympy} \PY{k+kn}{import} \PY{n}{integrate}\PY{p}{,} \PY{n}{diff}\PY{p}{,} \PY{n}{sqrt}\PY{p}{,} \PY{n}{cos}\PY{p}{,} \PY{n}{sin}\PY{p}{,} \PY{n}{pi}\PY{p}{,} \PY{n}{exp}\PY{p}{,} \PY{n}{log}
\PY{k+kn}{from} \PY{n+nn}{sympy}\PY{n+nn}{.}\PY{n+nn}{abc} \PY{k+kn}{import} \PY{o}{*} 
\PY{n}{i} \PY{o}{=} \PY{n}{sqrt}\PY{p}{(}\PY{o}{\PYZhy{}}\PY{l+m+mi}{1}\PY{p}{)}
\PY{k+kn}{import} \PY{n+nn}{numpy} \PY{k}{as} \PY{n+nn}{np}
\PY{k+kn}{import} \PY{n+nn}{sympy}\PY{n+nn}{.}\PY{n+nn}{printing} \PY{k}{as} \PY{n+nn}{printing}
\PY{n}{latp} \PY{o}{=} \PY{n}{printing}\PY{o}{.}\PY{n}{latex}
\PY{n}{hbar} \PY{o}{=} \PY{n}{sympy}\PY{o}{.}\PY{n}{symbols}\PY{p}{(}\PY{l+s+s2}{\PYZdq{}}\PY{l+s+s2}{hbar}\PY{l+s+s2}{\PYZdq{}}\PY{p}{,} \PY{n}{real}\PY{o}{=}\PY{k+kc}{True}\PY{p}{)}
\PY{k+kn}{import} \PY{n+nn}{mpmath} 
\PY{k+kn}{import} \PY{n+nn}{plotly}\PY{n+nn}{.}\PY{n+nn}{express} \PY{k}{as} \PY{n+nn}{px}
\PY{n}{hbar}
\end{Verbatim}
\end{tcolorbox}
 
            
\prompt{Out}{outcolor}{1}{}
    
    $\displaystyle \hbar$

    

    Where \(B\) is the spectral radiance and \(f\) is the frequency, we have
Planck's law of Blackbody Radiation. We will be using CGS units.

    \begin{tcolorbox}[breakable, size=fbox, boxrule=1pt, pad at break*=1mm,colback=cellbackground, colframe=cellborder]
\prompt{In}{incolor}{2}{\boxspacing}
\begin{Verbatim}[commandchars=\\\{\}]
\PY{k}{def} \PY{n+nf}{B}\PY{p}{(}\PY{n}{T}\PY{p}{,} \PY{n}{l}\PY{p}{,} \PY{n}{number} \PY{o}{=} \PY{k+kc}{False}\PY{p}{,} \PY{n}{dps}\PY{o}{=}\PY{l+m+mi}{50}\PY{p}{)}\PY{p}{:}
    \PY{k}{if} \PY{n}{number}\PY{p}{:}
        \PY{n}{mpmath}\PY{o}{.}\PY{n}{mp}\PY{o}{.}\PY{n}{dps} \PY{o}{=} \PY{n}{dps}
        \PY{n}{k\PYZus{}B} \PY{o}{=} \PY{n}{const}\PY{o}{.}\PY{n}{k\PYZus{}B}\PY{o}{.}\PY{n}{value}
        \PY{n}{h} \PY{o}{=} \PY{n}{const}\PY{o}{.}\PY{n}{h}\PY{o}{.}\PY{n}{value}
        \PY{n}{c} \PY{o}{=} \PY{n}{const}\PY{o}{.}\PY{n}{c}\PY{o}{.}\PY{n}{value}
        \PY{n}{T} \PY{o}{=} \PY{n}{mpmath}\PY{o}{.}\PY{n}{mpmathify}\PY{p}{(}\PY{n}{T}\PY{p}{)}
        \PY{k}{return} \PY{l+m+mi}{2}\PY{o}{*}\PY{n}{h}\PY{o}{*} \PY{p}{(} \PY{p}{(} \PY{n}{c} \PY{o}{/} \PY{n}{l} \PY{p}{)}\PY{o}{*}\PY{o}{*}\PY{l+m+mi}{3} \PY{p}{)} \PY{o}{/} \PY{p}{(}
            \PY{n}{c}\PY{o}{*}\PY{o}{*}\PY{l+m+mi}{2} \PY{o}{*} \PY{p}{(}\PY{n}{mpmath}\PY{o}{.}\PY{n}{exp}\PY{p}{(}\PY{n}{h} \PY{o}{*} \PY{n}{c}\PY{o}{/} \PY{p}{(}\PY{n}{l} \PY{o}{*} \PY{n}{k\PYZus{}B} \PY{o}{*} \PY{n}{T}\PY{p}{)}\PY{p}{)} \PY{o}{\PYZhy{}} \PY{l+m+mi}{1}\PY{p}{)}\PY{p}{)}
    \PY{k}{else}\PY{p}{:}
        \PY{n}{k\PYZus{}B}\PY{p}{,} \PY{n}{h}\PY{p}{,} \PY{n}{c} \PY{o}{=} \PY{n}{sympy}\PY{o}{.}\PY{n}{symbols}\PY{p}{(}\PY{l+s+s2}{\PYZdq{}}\PY{l+s+s2}{k\PYZus{}B h c}\PY{l+s+s2}{\PYZdq{}}\PY{p}{)}
        \PY{k}{return} \PY{l+m+mi}{2}\PY{o}{*}\PY{n}{h}\PY{o}{*} \PY{p}{(} \PY{p}{(} \PY{n}{c} \PY{o}{/} \PY{n}{l} \PY{p}{)}\PY{o}{*}\PY{o}{*}\PY{l+m+mi}{3} \PY{p}{)} \PY{o}{/} \PY{p}{(}
            \PY{n}{c}\PY{o}{*}\PY{o}{*}\PY{l+m+mi}{2} \PY{o}{*} \PY{p}{(}\PY{n}{exp}\PY{p}{(}\PY{n}{h} \PY{o}{*} \PY{n}{c}\PY{o}{/} \PY{p}{(}\PY{n}{l} \PY{o}{*} \PY{n}{k\PYZus{}B} \PY{o}{*} \PY{n}{T}\PY{p}{)}\PY{p}{)} \PY{o}{\PYZhy{}} \PY{l+m+mi}{1}\PY{p}{)}\PY{p}{)}
\PY{n}{md}\PY{p}{(}\PY{l+s+s2}{\PYZdq{}}\PY{l+s+s2}{We have the Planck Blackbody Law}\PY{l+s+s2}{\PYZdq{}} \PY{o}{+} \PY{l+s+s2}{\PYZdq{}}\PY{l+s+se}{\PYZbs{}\PYZbs{}}\PY{l+s+s2}{begin}\PY{l+s+si}{\PYZob{}equation\PYZcb{}}\PY{l+s+s2}{ B\PYZus{}l = }\PY{l+s+s2}{\PYZdq{}} \PY{o}{+} \PY{n}{latp}\PY{p}{(}\PY{n}{B}\PY{p}{(}\PY{n}{T}\PY{p}{,}\PY{n}{c}\PY{o}{/}\PY{n}{v}\PY{p}{)}\PY{p}{)}
   \PY{o}{+} \PY{l+s+s2}{\PYZdq{}}\PY{l+s+se}{\PYZbs{}\PYZbs{}}\PY{l+s+s2}{end}\PY{l+s+si}{\PYZob{}equation\PYZcb{}}\PY{l+s+s2}{\PYZdq{}}\PY{p}{)}
\end{Verbatim}
\end{tcolorbox}
 
            
\prompt{Out}{outcolor}{2}{}
    
    We have the Planck Blackbody
Law\begin{equation} B_l = \frac{2 h v^{3}}{c^{2} \left(e^{\frac{h v}{T k_{B}}} - 1\right)}\end{equation}

    

    \begin{tcolorbox}[breakable, size=fbox, boxrule=1pt, pad at break*=1mm,colback=cellbackground, colframe=cellborder]
\prompt{In}{incolor}{3}{\boxspacing}
\begin{Verbatim}[commandchars=\\\{\}]
\PY{k+kn}{from} \PY{n+nn}{sympy}\PY{n+nn}{.}\PY{n+nn}{plotting} \PY{k+kn}{import} \PY{n}{plot}
\PY{k}{def} \PY{n+nf}{plot\PYZus{}planck}\PY{p}{(}\PY{n}{T}\PY{p}{)}\PY{p}{:}
    \PY{n}{wavelengths} \PY{o}{=} \PY{n}{np}\PY{o}{.}\PY{n}{linspace}\PY{p}{(}\PY{l+m+mi}{350} \PY{o}{*} \PY{l+m+mi}{10}\PY{o}{*}\PY{o}{*}\PY{o}{\PYZhy{}}\PY{l+m+mi}{9}\PY{p}{,} \PY{l+m+mi}{1600} \PY{o}{*} \PY{l+m+mi}{10}\PY{o}{*}\PY{o}{*}\PY{o}{\PYZhy{}}\PY{l+m+mi}{9}\PY{p}{,} \PY{l+m+mi}{10}\PY{o}{*}\PY{o}{*}\PY{l+m+mi}{4}\PY{p}{)}
    \PY{n}{radiances} \PY{o}{=} \PY{p}{[}\PY{n+nb}{float}\PY{p}{(}\PY{n}{B}\PY{p}{(}\PY{n}{T}\PY{p}{,} \PY{n}{l}\PY{p}{,} \PY{k+kc}{True}\PY{p}{)}\PY{p}{)} \PY{k}{for} \PY{n}{l} \PY{o+ow}{in} \PY{n}{wavelengths}\PY{p}{]}
    \PY{n}{yname} \PY{o}{=}  \PY{l+s+s2}{\PYZdq{}}\PY{l+s+s2}{Spectral Radiance (W/(Hz sr m m))}\PY{l+s+s2}{\PYZdq{}}
    \PY{n}{df} \PY{o}{=} \PY{n}{pandas}\PY{o}{.}\PY{n}{DataFrame}\PY{p}{(}\PY{p}{[}\PY{n}{wavelengths}\PY{p}{,} \PY{n}{radiances}\PY{p}{]}\PY{p}{)}\PY{o}{.}\PY{n}{transpose}\PY{p}{(}\PY{p}{)}
    \PY{n}{df}\PY{o}{.}\PY{n}{columns} \PY{o}{=} \PY{p}{[}\PY{l+s+s2}{\PYZdq{}}\PY{l+s+s2}{Wavelength (m)}\PY{l+s+s2}{\PYZdq{}}\PY{p}{,} \PY{n}{yname}\PY{p}{]}
    \PY{n}{fig} \PY{o}{=} \PY{n}{px}\PY{o}{.}\PY{n}{scatter}\PY{p}{(}\PY{n}{df}\PY{p}{,} \PY{n}{x}\PY{o}{=}\PY{l+s+s2}{\PYZdq{}}\PY{l+s+s2}{Wavelength (m)}\PY{l+s+s2}{\PYZdq{}}\PY{p}{,} \PY{n}{y} \PY{o}{=} \PY{n}{yname}\PY{p}{,} 
                      \PY{n}{title} \PY{o}{=} \PY{l+s+s2}{\PYZdq{}}\PY{l+s+s2}{Planck Function for }\PY{l+s+s2}{\PYZdq{}} \PY{o}{+} \PY{n+nb}{str}\PY{p}{(}\PY{n}{T}\PY{p}{)} \PY{o}{+} \PY{l+s+s2}{\PYZdq{}}\PY{l+s+s2}{ Kelvin Ideal Blackbody}\PY{l+s+s2}{\PYZdq{}}\PY{p}{)}
    \PY{n}{fig}\PY{o}{.}\PY{n}{write\PYZus{}html}\PY{p}{(}\PY{n+nb}{str}\PY{p}{(}\PY{n}{T}\PY{p}{)} \PY{o}{+} \PY{l+s+s2}{\PYZdq{}}\PY{l+s+s2}{blackbody\PYZus{}plot.html}\PY{l+s+s2}{\PYZdq{}}\PY{p}{)}
    \PY{n}{plt}\PY{o}{.}\PY{n}{figure}\PY{p}{(}\PY{n}{dpi} \PY{o}{=} \PY{l+m+mi}{800}\PY{p}{)}
    \PY{n}{plt}\PY{o}{.}\PY{n}{scatter}\PY{p}{(}\PY{n}{wavelengths}\PY{p}{,} \PY{n}{radiances}\PY{p}{)}
    \PY{n}{plt}\PY{o}{.}\PY{n}{xlabel}\PY{p}{(}\PY{l+s+s2}{\PYZdq{}}\PY{l+s+s2}{Wavelength (m)}\PY{l+s+s2}{\PYZdq{}}\PY{p}{)}
    \PY{n}{plt}\PY{o}{.}\PY{n}{grid}\PY{p}{(}\PY{n}{which}\PY{o}{=}\PY{l+s+s1}{\PYZsq{}}\PY{l+s+s1}{minor}\PY{l+s+s1}{\PYZsq{}}\PY{p}{,} \PY{n}{axis}\PY{o}{=}\PY{l+s+s1}{\PYZsq{}}\PY{l+s+s1}{both}\PY{l+s+s1}{\PYZsq{}}\PY{p}{,} \PY{n}{visible}\PY{o}{=}\PY{k+kc}{True}\PY{p}{)}
    \PY{n}{plt}\PY{o}{.}\PY{n}{grid}\PY{p}{(}\PY{n}{which}\PY{o}{=}\PY{l+s+s1}{\PYZsq{}}\PY{l+s+s1}{major}\PY{l+s+s1}{\PYZsq{}}\PY{p}{,} \PY{n}{axis}\PY{o}{=}\PY{l+s+s1}{\PYZsq{}}\PY{l+s+s1}{both}\PY{l+s+s1}{\PYZsq{}}\PY{p}{,} \PY{n}{visible}\PY{o}{=}\PY{k+kc}{True}\PY{p}{)}
    \PY{n}{plt}\PY{o}{.}\PY{n}{ylabel}\PY{p}{(}\PY{l+s+s2}{\PYZdq{}}\PY{l+s+s2}{Spectral Radiance (W Hz\PYZdl{}\PYZca{}}\PY{l+s+s2}{\PYZob{}}\PY{l+s+s2}{\PYZhy{}1\PYZcb{}\PYZdl{} sr\PYZdl{}\PYZca{}}\PY{l+s+s2}{\PYZob{}}\PY{l+s+s2}{\PYZhy{}1\PYZcb{}\PYZdl{} m\PYZdl{}\PYZca{}2\PYZdl{})}\PY{l+s+s2}{\PYZdq{}}\PY{p}{)} 
    \PY{n}{plt}\PY{o}{.}\PY{n}{title}\PY{p}{(}\PY{n+nb}{str}\PY{p}{(}\PY{n}{T}\PY{p}{)} \PY{o}{+} \PY{l+s+s2}{\PYZdq{}}\PY{l+s+s2}{ Kelvin Ideal Blackbody}\PY{l+s+s2}{\PYZdq{}}\PY{p}{)}
    \PY{k}{return} \PY{n}{plt}
    \PY{l+s+sd}{\PYZsq{}\PYZsq{}\PYZsq{}}
\PY{l+s+sd}{    plot(b\PYZus{}fun, xlim = (350 * 10**\PYZhy{}9, 1600 * 10**\PYZhy{}9), ylim=(.9,10),}
\PY{l+s+sd}{        adaptive = False, nb\PYZus{}of\PYZus{}points = 10**8, yscale=\PYZdq{}log\PYZdq{}, xlabel = \PYZdq{}Wavelength (nm)\PYZdq{},}
\PY{l+s+sd}{        title = \PYZdq{}Planck Function for Blackbody at \PYZdq{} + str(T) + \PYZdq{} Kelvin\PYZdq{},}
\PY{l+s+sd}{        ylabel = \PYZdq{}Spectral Radiance (W Hz\PYZdl{}\PYZca{}\PYZob{}\PYZhy{}1\PYZcb{}\PYZdl{} sr\PYZdl{}\PYZca{}\PYZob{}\PYZhy{}1\PYZcb{}\PYZdl{} m\PYZdl{}\PYZca{}2\PYZdl{})\PYZdq{})}
\PY{l+s+sd}{    \PYZsq{}\PYZsq{}\PYZsq{}}
\end{Verbatim}
\end{tcolorbox}

    \begin{tcolorbox}[breakable, size=fbox, boxrule=1pt, pad at break*=1mm,colback=cellbackground, colframe=cellborder]
\prompt{In}{incolor}{4}{\boxspacing}
\begin{Verbatim}[commandchars=\\\{\}]
\PY{n}{plot\PYZus{}planck}\PY{p}{(}\PY{l+m+mi}{300}\PY{p}{)}
\end{Verbatim}
\end{tcolorbox}

            \begin{tcolorbox}[breakable, size=fbox, boxrule=.5pt, pad at break*=1mm, opacityfill=0]
\prompt{Out}{outcolor}{4}{\boxspacing}
\begin{Verbatim}[commandchars=\\\{\}]
<module 'matplotlib.pyplot' from 'C:\textbackslash{}\textbackslash{}Users\textbackslash{}\textbackslash{}marco\textbackslash{}\textbackslash{}anaconda3\textbackslash{}\textbackslash{}lib\textbackslash{}\textbackslash{}site-
packages\textbackslash{}\textbackslash{}matplotlib\textbackslash{}\textbackslash{}pyplot.py'>
\end{Verbatim}
\end{tcolorbox}
        
    \begin{center}
    \adjustimage{max size={0.9\linewidth}{0.9\paperheight}}{Lab1_files/Lab1_4_1.png}
    \end{center}
    { \hspace*{\fill} \\}
    
    \begin{tcolorbox}[breakable, size=fbox, boxrule=1pt, pad at break*=1mm,colback=cellbackground, colframe=cellborder]
\prompt{In}{incolor}{5}{\boxspacing}
\begin{Verbatim}[commandchars=\\\{\}]
\PY{n}{plot\PYZus{}planck}\PY{p}{(}\PY{l+m+mi}{1000}\PY{p}{)}
\end{Verbatim}
\end{tcolorbox}

            \begin{tcolorbox}[breakable, size=fbox, boxrule=.5pt, pad at break*=1mm, opacityfill=0]
\prompt{Out}{outcolor}{5}{\boxspacing}
\begin{Verbatim}[commandchars=\\\{\}]
<module 'matplotlib.pyplot' from 'C:\textbackslash{}\textbackslash{}Users\textbackslash{}\textbackslash{}marco\textbackslash{}\textbackslash{}anaconda3\textbackslash{}\textbackslash{}lib\textbackslash{}\textbackslash{}site-
packages\textbackslash{}\textbackslash{}matplotlib\textbackslash{}\textbackslash{}pyplot.py'>
\end{Verbatim}
\end{tcolorbox}
        
    \begin{center}
    \adjustimage{max size={0.9\linewidth}{0.9\paperheight}}{Lab1_files/Lab1_5_1.png}
    \end{center}
    { \hspace*{\fill} \\}
    
    \begin{tcolorbox}[breakable, size=fbox, boxrule=1pt, pad at break*=1mm,colback=cellbackground, colframe=cellborder]
\prompt{In}{incolor}{6}{\boxspacing}
\begin{Verbatim}[commandchars=\\\{\}]
\PY{n}{plot\PYZus{}planck}\PY{p}{(}\PY{l+m+mi}{3000}\PY{p}{)}
\end{Verbatim}
\end{tcolorbox}

            \begin{tcolorbox}[breakable, size=fbox, boxrule=.5pt, pad at break*=1mm, opacityfill=0]
\prompt{Out}{outcolor}{6}{\boxspacing}
\begin{Verbatim}[commandchars=\\\{\}]
<module 'matplotlib.pyplot' from 'C:\textbackslash{}\textbackslash{}Users\textbackslash{}\textbackslash{}marco\textbackslash{}\textbackslash{}anaconda3\textbackslash{}\textbackslash{}lib\textbackslash{}\textbackslash{}site-
packages\textbackslash{}\textbackslash{}matplotlib\textbackslash{}\textbackslash{}pyplot.py'>
\end{Verbatim}
\end{tcolorbox}
        
    \begin{center}
    \adjustimage{max size={0.9\linewidth}{0.9\paperheight}}{Lab1_files/Lab1_6_1.png}
    \end{center}
    { \hspace*{\fill} \\}
    
    \begin{tcolorbox}[breakable, size=fbox, boxrule=1pt, pad at break*=1mm,colback=cellbackground, colframe=cellborder]
\prompt{In}{incolor}{7}{\boxspacing}
\begin{Verbatim}[commandchars=\\\{\}]
\PY{n}{plot\PYZus{}planck}\PY{p}{(}\PY{l+m+mi}{5000}\PY{p}{)}
\end{Verbatim}
\end{tcolorbox}

            \begin{tcolorbox}[breakable, size=fbox, boxrule=.5pt, pad at break*=1mm, opacityfill=0]
\prompt{Out}{outcolor}{7}{\boxspacing}
\begin{Verbatim}[commandchars=\\\{\}]
<module 'matplotlib.pyplot' from 'C:\textbackslash{}\textbackslash{}Users\textbackslash{}\textbackslash{}marco\textbackslash{}\textbackslash{}anaconda3\textbackslash{}\textbackslash{}lib\textbackslash{}\textbackslash{}site-
packages\textbackslash{}\textbackslash{}matplotlib\textbackslash{}\textbackslash{}pyplot.py'>
\end{Verbatim}
\end{tcolorbox}
        
    \begin{center}
    \adjustimage{max size={0.9\linewidth}{0.9\paperheight}}{Lab1_files/Lab1_7_1.png}
    \end{center}
    { \hspace*{\fill} \\}
    
    \begin{tcolorbox}[breakable, size=fbox, boxrule=1pt, pad at break*=1mm,colback=cellbackground, colframe=cellborder]
\prompt{In}{incolor}{8}{\boxspacing}
\begin{Verbatim}[commandchars=\\\{\}]
\PY{n}{temp\PYZus{}list} \PY{o}{=} \PY{p}{[}\PY{l+m+mi}{300}\PY{p}{,} \PY{l+m+mi}{1000}\PY{p}{,} \PY{l+m+mi}{3000}\PY{p}{,} \PY{l+m+mi}{5000}\PY{p}{]}
\PY{k}{for} \PY{n}{T} \PY{o+ow}{in} \PY{n}{temp\PYZus{}list}\PY{p}{:}
    \PY{n}{peak} \PY{o}{=} \PY{p}{(}\PY{l+m+mi}{3} \PY{o}{*} \PY{n}{units}\PY{o}{.}\PY{n}{mm} \PY{o}{/} \PY{n}{T}\PY{p}{)}\PY{o}{.}\PY{n}{to}\PY{p}{(}\PY{n}{units}\PY{o}{.}\PY{n}{m}\PY{p}{)}\PY{o}{.}\PY{n}{value}
    \PY{n+nb}{print}\PY{p}{(}\PY{n}{mpmath}\PY{o}{.}\PY{n}{log10}\PY{p}{(}\PY{n}{peak}\PY{p}{)}\PY{p}{,} \PY{n}{mpmath}\PY{o}{.}\PY{n}{log10}\PY{p}{(}\PY{n}{B}\PY{p}{(}\PY{n}{T}\PY{p}{,} \PY{n}{peak}\PY{p}{,} \PY{k+kc}{True}\PY{p}{)}\PY{p}{)}\PY{p}{)}
\end{Verbatim}
\end{tcolorbox}

    \begin{Verbatim}[commandchars=\\\{\}]
-4.9999999999999999644733850823019973008295631257525
-11.480132488738328539155182878367842691858105964728
-5.5228787452803375517024728445662672420376910518749
-9.9114962528973159081789007475600640111092417400786
-6.0000000000000000196526453297206160616919506824173
-8.4801324887383286630048454686219256681457103788396
-6.2218487496163563871438785317410077257236319367645
-7.8145862398892592686797972725455481807151190300996
    \end{Verbatim}

    \section{Aprill 11, 2023}

Recording data here to be printed out later and turned in with the
report.

\underline{Lecture Notes} Notebook management and safety! Lab 1: as
expected from the materials, carefully measure the output from a
blackbody with a photodiode. 211 is bad luck for projecting. 8
Labs/exercises total which will be in Room B129 after a short lecture in
211. Emphasis on good notebook practice. Final lab notebook due at the
end of the course. 10\% grade on intro lecture, 50\% based on notebook
and reports. 40\% based on the lab itself. Data can be shared within the
group but the write-up must be your own. Must be clear and coherent, but
not necessarily polished. You can email Emily the TA via email on
Mondays at 5PM.

\subsection{Notebook Guidelines}

Introduction, Equipment (e.g.~spacing and the distance of the collimated
lens and filter from the photodiode), Measurements (incorporating data
error where appropriate), Data Analysis (including any plots!!! today
we're measuring the spectral properties of a quartz lamp),
Interpretation and Discussion (why did it all go wrong? Understand why
the measurements did or did not agree with expectation), Summary and
Conclusion (map back to introduction! we set up this halogen lamp and
the photodiode, and it did or did not match expectation for these
reasons).

This week we're using a quartz lamp, a set of filters, and will measure
the photoinsity of the lamp with a photodiode as a function of
wavelength to determine if it's a blackbody spectrum. Describe what
equipment was set up and why, sketch the layout, and don't just
transcribe the lab handout. Record what you actually used in the lab
environment. The interference filter assumes light in collimated space!
Use a scale of the operation of the optics and why. Christoff Bernick
works on optical instrumentation in Hawaii. Indicate uncertainty
measurements whenever possible and reasonable!! Check your measured
values against expectations! Describe any problems that occured (you
won't get penalized for these specifically)! Don't erase anything (you
might have been right the first time!). Show tables and plots where
appropriate. Here, the photointensity will be a function of wavelength
and angle. Pay attention to significant figures when appropriate. You
should be spending no more than 3-4 hours writing up the report. Include
the code written to plot things, and any information that helped
determine the signal to noise can be included. What do your results
mean? Are they consistent with your explanation, and why or why not?
This will be the bulk of the report. How could the lab be improved? What
did you learn? Interpretation and discussion is to be done at home.
Summarize your results rather than your expectation.

Record everything you do! Overhead for thoroughness is always worth it.
It makes the process of decoding what you've done that much easier. It
must be easily reproducible by someone less experienced who has not done
it yet. Be complete, clear, and organized. Each person turns in their
own complete data lab notebook. Discussion in the lab is perfectly fine.
The person recording data can photocopy it and share.

Today we are focused on optical alignment in 3 dimensions and how to
work on an optical table. Hopefully the lamp gives you a collimated beam
that goes through the optical filter. Put back everything exactly where
you found it. Dimitri redid the course and modernized it while replacing
all of the equipment. Keep it that way.

Don't touch the optics and wear gloves.

Will rotate the filters and write the measured number of photons coming
through the filter of the photodiode. Need to know the stability of the
lamp, are you on axis, how large is the iris, what shape is the filter?
(it will have some shape characteristic of its materials). All of the
materials and will be neatly outlined. Some of the equipment is on the
bench and some of it is elsewhere in the lab. Set up your lamp first and
follow the outline in the handout for the rest of the equipment, one at
a time.

Will work in two groups in two benches. The groups can change. Lens
maker's equations! Plot power as a function of distance two. Will send
materials just in case.

    Lab notes:

\begin{itemize}
\item 2:40 PM- mounted the translational stage (xyz) onto the table with the z stage facing towards the quartz lamp. It is not yet screwed in but appears to be centered by the hole. The iris has been firmly mounted in place, and so has the lamp. The lamp and iris appear vertically aligned. See slide 1 in the slideshow (https://docs.google.com/presentation/d/1V9HHvC10PpjYFTfDer1Gq4PeY7kGloljpdjYYd0YHjA/edit#slide=id.p) We will use a paper test to perfect the alignment. Our intent is to only take measurements at the peak power (as a function of distance, orientation, but we mostly care about wavelength) such that we have an absolute reference and do not need to repeat the entire experiment should something go awry. Will be using the filters and the lenses in the rotating ring to align and mount. 
\item Nevermind we do have something to mount the lens so it won't go on the wheel, but we will put it on the wheel regardless. 
\item just mounted the xyz stage after a paper test that Tanmay took a picture of (see the fringes on slide 2). Riley and Jonah are now screwing in and further aligning the lens with the optical train (slide 3). Now aligning the optical train immediately after. I mounted photodiode to the optical train.
Carlos is inserting the filters into the filter wheel. Tanmay is going between tasks for thorough collaboration. 
\item Remounting the lamp to align in the z direction. Tanmay did in fact use a washer. Decreased the size of the iris to confirm alignment (see small iris, high on the power meter we just connected) in slide 5 and how it's lower on slide 6 with it off
\item AMBIENT LIGHT WILL HAVE A SPECTRUM FROM THE LAMP. DO READINGS WITH THE LIGHT OFF TOO. -1.76 nA with the cap on. -2.5 nA now with the cap on. Subtract the errors in quadrature. now at -3.4 nA with the cap off (I think the cap is just cooling down). Click $\Delta$ to zero the power meter. 
\item 3:26 PM. Now closing and about to 0 it. Just zeroed it. Now it's at .4 nA a few seconds after zeroing. The instrument uncertainty is in the datasheet. 
\item Measuring from the middle of the lens, we are 2 cm further than we should be. Will now translate the detector to put it in the focal point. 
\item 3:29 PM. Now moving the iris vertically. Appeared to have been unaligned before. 
\item Add the noise in quadrature. Don't force anything. Nothing here should require more force than a few fingers or move quickly or crash and bang. Due to chromatic aberration, each filter will have a different focus point. 
\item Varies by approximatly 70 nA from open to maximally closed iris. Will use a consistent aperture for each measurement. 
\item Each filter increases from 2-6 increasing in wavelength. Number 2 is 350 nm, number 3 is 400 nm, number 4 is 450 nm. They increase by 50 nm each. The last one is 600 nm (currently in the wheel). 
\item Currently 27 cm from the lens.
\item 3:41 PM. Tanmay aligned the iris. 
\item 3:46 PM. zeroing with the ambient light.
\item 350 nm: just zeroed and moved the photodiode forward again. Now at 126 nA. There is less 1/r squared dependence when it is collimated.  


\end{itemize}

    \begin{tcolorbox}[breakable, size=fbox, boxrule=1pt, pad at break*=1mm,colback=cellbackground, colframe=cellborder]
\prompt{In}{incolor}{ }{\boxspacing}
\begin{Verbatim}[commandchars=\\\{\}]
\PY{o}{!}jupyter nbconvert \PYZhy{}\PYZhy{}to latex Lab1.ipynb
\end{Verbatim}
\end{tcolorbox}

    \begin{tcolorbox}[breakable, size=fbox, boxrule=1pt, pad at break*=1mm,colback=cellbackground, colframe=cellborder]
\prompt{In}{incolor}{ }{\boxspacing}
\begin{Verbatim}[commandchars=\\\{\}]

\end{Verbatim}
\end{tcolorbox}


    % Add a bibliography block to the postdoc
    
    
    
\end{document}
