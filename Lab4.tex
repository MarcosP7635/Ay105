\documentclass[11pt]{article}

    \usepackage[breakable]{tcolorbox}
    \usepackage{parskip} % Stop auto-indenting (to mimic markdown behaviour)
    

    % Basic figure setup, for now with no caption control since it's done
    % automatically by Pandoc (which extracts ![](path) syntax from Markdown).
    \usepackage{graphicx}
    % Maintain compatibility with old templates. Remove in nbconvert 6.0
    \let\Oldincludegraphics\includegraphics
    % Ensure that by default, figures have no caption (until we provide a
    % proper Figure object with a Caption API and a way to capture that
    % in the conversion process - todo).
    \usepackage{caption}
    \DeclareCaptionFormat{nocaption}{}
    \captionsetup{format=nocaption,aboveskip=0pt,belowskip=0pt}

    \usepackage{float}
    \floatplacement{figure}{H} % forces figures to be placed at the correct location
    \usepackage{xcolor} % Allow colors to be defined
    \usepackage{enumerate} % Needed for markdown enumerations to work
    \usepackage{geometry} % Used to adjust the document margins
    \usepackage{amsmath} % Equations
    \usepackage{amssymb} % Equations
    \usepackage{textcomp} % defines textquotesingle
    % Hack from http://tex.stackexchange.com/a/47451/13684:
    \AtBeginDocument{%
        \def\PYZsq{\textquotesingle}% Upright quotes in Pygmentized code
    }
    \usepackage{upquote} % Upright quotes for verbatim code
    \usepackage{eurosym} % defines \euro

    \usepackage{iftex}
    \ifPDFTeX
        \usepackage[T1]{fontenc}
        \IfFileExists{alphabeta.sty}{
              \usepackage{alphabeta}
          }{
              \usepackage[mathletters]{ucs}
              \usepackage[utf8x]{inputenc}
          }
    \else
        \usepackage{fontspec}
        \usepackage{unicode-math}
    \fi

    \usepackage{fancyvrb} % verbatim replacement that allows latex
    \usepackage{grffile} % extends the file name processing of package graphics 
                         % to support a larger range
    \makeatletter % fix for old versions of grffile with XeLaTeX
    \@ifpackagelater{grffile}{2019/11/01}
    {
      % Do nothing on new versions
    }
    {
      \def\Gread@@xetex#1{%
        \IfFileExists{"\Gin@base".bb}%
        {\Gread@eps{\Gin@base.bb}}%
        {\Gread@@xetex@aux#1}%
      }
    }
    \makeatother
    \usepackage[Export]{adjustbox} % Used to constrain images to a maximum size
    \adjustboxset{max size={0.9\linewidth}{0.9\paperheight}}

    % The hyperref package gives us a pdf with properly built
    % internal navigation ('pdf bookmarks' for the table of contents,
    % internal cross-reference links, web links for URLs, etc.)
    \usepackage{hyperref}
    % The default LaTeX title has an obnoxious amount of whitespace. By default,
    % titling removes some of it. It also provides customization options.
    \usepackage{titling}
    \usepackage{longtable} % longtable support required by pandoc >1.10
    \usepackage{booktabs}  % table support for pandoc > 1.12.2
    \usepackage{array}     % table support for pandoc >= 2.11.3
    \usepackage{calc}      % table minipage width calculation for pandoc >= 2.11.1
    \usepackage[inline]{enumitem} % IRkernel/repr support (it uses the enumerate* environment)
    \usepackage[normalem]{ulem} % ulem is needed to support strikethroughs (\sout)
                                % normalem makes italics be italics, not underlines
    \usepackage{mathrsfs}
    

    
    % Colors for the hyperref package
    \definecolor{urlcolor}{rgb}{0,.145,.698}
    \definecolor{linkcolor}{rgb}{.71,0.21,0.01}
    \definecolor{citecolor}{rgb}{.12,.54,.11}

    % ANSI colors
    \definecolor{ansi-black}{HTML}{3E424D}
    \definecolor{ansi-black-intense}{HTML}{282C36}
    \definecolor{ansi-red}{HTML}{E75C58}
    \definecolor{ansi-red-intense}{HTML}{B22B31}
    \definecolor{ansi-green}{HTML}{00A250}
    \definecolor{ansi-green-intense}{HTML}{007427}
    \definecolor{ansi-yellow}{HTML}{DDB62B}
    \definecolor{ansi-yellow-intense}{HTML}{B27D12}
    \definecolor{ansi-blue}{HTML}{208FFB}
    \definecolor{ansi-blue-intense}{HTML}{0065CA}
    \definecolor{ansi-magenta}{HTML}{D160C4}
    \definecolor{ansi-magenta-intense}{HTML}{A03196}
    \definecolor{ansi-cyan}{HTML}{60C6C8}
    \definecolor{ansi-cyan-intense}{HTML}{258F8F}
    \definecolor{ansi-white}{HTML}{C5C1B4}
    \definecolor{ansi-white-intense}{HTML}{A1A6B2}
    \definecolor{ansi-default-inverse-fg}{HTML}{FFFFFF}
    \definecolor{ansi-default-inverse-bg}{HTML}{000000}

    % common color for the border for error outputs.
    \definecolor{outerrorbackground}{HTML}{FFDFDF}

    % commands and environments needed by pandoc snippets
    % extracted from the output of `pandoc -s`
    \providecommand{\tightlist}{%
      \setlength{\itemsep}{0pt}\setlength{\parskip}{0pt}}
    \DefineVerbatimEnvironment{Highlighting}{Verbatim}{commandchars=\\\{\}}
    % Add ',fontsize=\small' for more characters per line
    \newenvironment{Shaded}{}{}
    \newcommand{\KeywordTok}[1]{\textcolor[rgb]{0.00,0.44,0.13}{\textbf{{#1}}}}
    \newcommand{\DataTypeTok}[1]{\textcolor[rgb]{0.56,0.13,0.00}{{#1}}}
    \newcommand{\DecValTok}[1]{\textcolor[rgb]{0.25,0.63,0.44}{{#1}}}
    \newcommand{\BaseNTok}[1]{\textcolor[rgb]{0.25,0.63,0.44}{{#1}}}
    \newcommand{\FloatTok}[1]{\textcolor[rgb]{0.25,0.63,0.44}{{#1}}}
    \newcommand{\CharTok}[1]{\textcolor[rgb]{0.25,0.44,0.63}{{#1}}}
    \newcommand{\StringTok}[1]{\textcolor[rgb]{0.25,0.44,0.63}{{#1}}}
    \newcommand{\CommentTok}[1]{\textcolor[rgb]{0.38,0.63,0.69}{\textit{{#1}}}}
    \newcommand{\OtherTok}[1]{\textcolor[rgb]{0.00,0.44,0.13}{{#1}}}
    \newcommand{\AlertTok}[1]{\textcolor[rgb]{1.00,0.00,0.00}{\textbf{{#1}}}}
    \newcommand{\FunctionTok}[1]{\textcolor[rgb]{0.02,0.16,0.49}{{#1}}}
    \newcommand{\RegionMarkerTok}[1]{{#1}}
    \newcommand{\ErrorTok}[1]{\textcolor[rgb]{1.00,0.00,0.00}{\textbf{{#1}}}}
    \newcommand{\NormalTok}[1]{{#1}}
    
    % Additional commands for more recent versions of Pandoc
    \newcommand{\ConstantTok}[1]{\textcolor[rgb]{0.53,0.00,0.00}{{#1}}}
    \newcommand{\SpecialCharTok}[1]{\textcolor[rgb]{0.25,0.44,0.63}{{#1}}}
    \newcommand{\VerbatimStringTok}[1]{\textcolor[rgb]{0.25,0.44,0.63}{{#1}}}
    \newcommand{\SpecialStringTok}[1]{\textcolor[rgb]{0.73,0.40,0.53}{{#1}}}
    \newcommand{\ImportTok}[1]{{#1}}
    \newcommand{\DocumentationTok}[1]{\textcolor[rgb]{0.73,0.13,0.13}{\textit{{#1}}}}
    \newcommand{\AnnotationTok}[1]{\textcolor[rgb]{0.38,0.63,0.69}{\textbf{\textit{{#1}}}}}
    \newcommand{\CommentVarTok}[1]{\textcolor[rgb]{0.38,0.63,0.69}{\textbf{\textit{{#1}}}}}
    \newcommand{\VariableTok}[1]{\textcolor[rgb]{0.10,0.09,0.49}{{#1}}}
    \newcommand{\ControlFlowTok}[1]{\textcolor[rgb]{0.00,0.44,0.13}{\textbf{{#1}}}}
    \newcommand{\OperatorTok}[1]{\textcolor[rgb]{0.40,0.40,0.40}{{#1}}}
    \newcommand{\BuiltInTok}[1]{{#1}}
    \newcommand{\ExtensionTok}[1]{{#1}}
    \newcommand{\PreprocessorTok}[1]{\textcolor[rgb]{0.74,0.48,0.00}{{#1}}}
    \newcommand{\AttributeTok}[1]{\textcolor[rgb]{0.49,0.56,0.16}{{#1}}}
    \newcommand{\InformationTok}[1]{\textcolor[rgb]{0.38,0.63,0.69}{\textbf{\textit{{#1}}}}}
    \newcommand{\WarningTok}[1]{\textcolor[rgb]{0.38,0.63,0.69}{\textbf{\textit{{#1}}}}}
    
    
    % Define a nice break command that doesn't care if a line doesn't already
    % exist.
    \def\br{\hspace*{\fill} \\* }
    % Math Jax compatibility definitions
    \def\gt{>}
    \def\lt{<}
    \let\Oldtex\TeX
    \let\Oldlatex\LaTeX
    \renewcommand{\TeX}{\textrm{\Oldtex}}
    \renewcommand{\LaTeX}{\textrm{\Oldlatex}}
    % Document parameters
    % Document title
    \title{Lab4}
    
    
    
    
    
% Pygments definitions
\makeatletter
\def\PY@reset{\let\PY@it=\relax \let\PY@bf=\relax%
    \let\PY@ul=\relax \let\PY@tc=\relax%
    \let\PY@bc=\relax \let\PY@ff=\relax}
\def\PY@tok#1{\csname PY@tok@#1\endcsname}
\def\PY@toks#1+{\ifx\relax#1\empty\else%
    \PY@tok{#1}\expandafter\PY@toks\fi}
\def\PY@do#1{\PY@bc{\PY@tc{\PY@ul{%
    \PY@it{\PY@bf{\PY@ff{#1}}}}}}}
\def\PY#1#2{\PY@reset\PY@toks#1+\relax+\PY@do{#2}}

\@namedef{PY@tok@w}{\def\PY@tc##1{\textcolor[rgb]{0.73,0.73,0.73}{##1}}}
\@namedef{PY@tok@c}{\let\PY@it=\textit\def\PY@tc##1{\textcolor[rgb]{0.24,0.48,0.48}{##1}}}
\@namedef{PY@tok@cp}{\def\PY@tc##1{\textcolor[rgb]{0.61,0.40,0.00}{##1}}}
\@namedef{PY@tok@k}{\let\PY@bf=\textbf\def\PY@tc##1{\textcolor[rgb]{0.00,0.50,0.00}{##1}}}
\@namedef{PY@tok@kp}{\def\PY@tc##1{\textcolor[rgb]{0.00,0.50,0.00}{##1}}}
\@namedef{PY@tok@kt}{\def\PY@tc##1{\textcolor[rgb]{0.69,0.00,0.25}{##1}}}
\@namedef{PY@tok@o}{\def\PY@tc##1{\textcolor[rgb]{0.40,0.40,0.40}{##1}}}
\@namedef{PY@tok@ow}{\let\PY@bf=\textbf\def\PY@tc##1{\textcolor[rgb]{0.67,0.13,1.00}{##1}}}
\@namedef{PY@tok@nb}{\def\PY@tc##1{\textcolor[rgb]{0.00,0.50,0.00}{##1}}}
\@namedef{PY@tok@nf}{\def\PY@tc##1{\textcolor[rgb]{0.00,0.00,1.00}{##1}}}
\@namedef{PY@tok@nc}{\let\PY@bf=\textbf\def\PY@tc##1{\textcolor[rgb]{0.00,0.00,1.00}{##1}}}
\@namedef{PY@tok@nn}{\let\PY@bf=\textbf\def\PY@tc##1{\textcolor[rgb]{0.00,0.00,1.00}{##1}}}
\@namedef{PY@tok@ne}{\let\PY@bf=\textbf\def\PY@tc##1{\textcolor[rgb]{0.80,0.25,0.22}{##1}}}
\@namedef{PY@tok@nv}{\def\PY@tc##1{\textcolor[rgb]{0.10,0.09,0.49}{##1}}}
\@namedef{PY@tok@no}{\def\PY@tc##1{\textcolor[rgb]{0.53,0.00,0.00}{##1}}}
\@namedef{PY@tok@nl}{\def\PY@tc##1{\textcolor[rgb]{0.46,0.46,0.00}{##1}}}
\@namedef{PY@tok@ni}{\let\PY@bf=\textbf\def\PY@tc##1{\textcolor[rgb]{0.44,0.44,0.44}{##1}}}
\@namedef{PY@tok@na}{\def\PY@tc##1{\textcolor[rgb]{0.41,0.47,0.13}{##1}}}
\@namedef{PY@tok@nt}{\let\PY@bf=\textbf\def\PY@tc##1{\textcolor[rgb]{0.00,0.50,0.00}{##1}}}
\@namedef{PY@tok@nd}{\def\PY@tc##1{\textcolor[rgb]{0.67,0.13,1.00}{##1}}}
\@namedef{PY@tok@s}{\def\PY@tc##1{\textcolor[rgb]{0.73,0.13,0.13}{##1}}}
\@namedef{PY@tok@sd}{\let\PY@it=\textit\def\PY@tc##1{\textcolor[rgb]{0.73,0.13,0.13}{##1}}}
\@namedef{PY@tok@si}{\let\PY@bf=\textbf\def\PY@tc##1{\textcolor[rgb]{0.64,0.35,0.47}{##1}}}
\@namedef{PY@tok@se}{\let\PY@bf=\textbf\def\PY@tc##1{\textcolor[rgb]{0.67,0.36,0.12}{##1}}}
\@namedef{PY@tok@sr}{\def\PY@tc##1{\textcolor[rgb]{0.64,0.35,0.47}{##1}}}
\@namedef{PY@tok@ss}{\def\PY@tc##1{\textcolor[rgb]{0.10,0.09,0.49}{##1}}}
\@namedef{PY@tok@sx}{\def\PY@tc##1{\textcolor[rgb]{0.00,0.50,0.00}{##1}}}
\@namedef{PY@tok@m}{\def\PY@tc##1{\textcolor[rgb]{0.40,0.40,0.40}{##1}}}
\@namedef{PY@tok@gh}{\let\PY@bf=\textbf\def\PY@tc##1{\textcolor[rgb]{0.00,0.00,0.50}{##1}}}
\@namedef{PY@tok@gu}{\let\PY@bf=\textbf\def\PY@tc##1{\textcolor[rgb]{0.50,0.00,0.50}{##1}}}
\@namedef{PY@tok@gd}{\def\PY@tc##1{\textcolor[rgb]{0.63,0.00,0.00}{##1}}}
\@namedef{PY@tok@gi}{\def\PY@tc##1{\textcolor[rgb]{0.00,0.52,0.00}{##1}}}
\@namedef{PY@tok@gr}{\def\PY@tc##1{\textcolor[rgb]{0.89,0.00,0.00}{##1}}}
\@namedef{PY@tok@ge}{\let\PY@it=\textit}
\@namedef{PY@tok@gs}{\let\PY@bf=\textbf}
\@namedef{PY@tok@gp}{\let\PY@bf=\textbf\def\PY@tc##1{\textcolor[rgb]{0.00,0.00,0.50}{##1}}}
\@namedef{PY@tok@go}{\def\PY@tc##1{\textcolor[rgb]{0.44,0.44,0.44}{##1}}}
\@namedef{PY@tok@gt}{\def\PY@tc##1{\textcolor[rgb]{0.00,0.27,0.87}{##1}}}
\@namedef{PY@tok@err}{\def\PY@bc##1{{\setlength{\fboxsep}{\string -\fboxrule}\fcolorbox[rgb]{1.00,0.00,0.00}{1,1,1}{\strut ##1}}}}
\@namedef{PY@tok@kc}{\let\PY@bf=\textbf\def\PY@tc##1{\textcolor[rgb]{0.00,0.50,0.00}{##1}}}
\@namedef{PY@tok@kd}{\let\PY@bf=\textbf\def\PY@tc##1{\textcolor[rgb]{0.00,0.50,0.00}{##1}}}
\@namedef{PY@tok@kn}{\let\PY@bf=\textbf\def\PY@tc##1{\textcolor[rgb]{0.00,0.50,0.00}{##1}}}
\@namedef{PY@tok@kr}{\let\PY@bf=\textbf\def\PY@tc##1{\textcolor[rgb]{0.00,0.50,0.00}{##1}}}
\@namedef{PY@tok@bp}{\def\PY@tc##1{\textcolor[rgb]{0.00,0.50,0.00}{##1}}}
\@namedef{PY@tok@fm}{\def\PY@tc##1{\textcolor[rgb]{0.00,0.00,1.00}{##1}}}
\@namedef{PY@tok@vc}{\def\PY@tc##1{\textcolor[rgb]{0.10,0.09,0.49}{##1}}}
\@namedef{PY@tok@vg}{\def\PY@tc##1{\textcolor[rgb]{0.10,0.09,0.49}{##1}}}
\@namedef{PY@tok@vi}{\def\PY@tc##1{\textcolor[rgb]{0.10,0.09,0.49}{##1}}}
\@namedef{PY@tok@vm}{\def\PY@tc##1{\textcolor[rgb]{0.10,0.09,0.49}{##1}}}
\@namedef{PY@tok@sa}{\def\PY@tc##1{\textcolor[rgb]{0.73,0.13,0.13}{##1}}}
\@namedef{PY@tok@sb}{\def\PY@tc##1{\textcolor[rgb]{0.73,0.13,0.13}{##1}}}
\@namedef{PY@tok@sc}{\def\PY@tc##1{\textcolor[rgb]{0.73,0.13,0.13}{##1}}}
\@namedef{PY@tok@dl}{\def\PY@tc##1{\textcolor[rgb]{0.73,0.13,0.13}{##1}}}
\@namedef{PY@tok@s2}{\def\PY@tc##1{\textcolor[rgb]{0.73,0.13,0.13}{##1}}}
\@namedef{PY@tok@sh}{\def\PY@tc##1{\textcolor[rgb]{0.73,0.13,0.13}{##1}}}
\@namedef{PY@tok@s1}{\def\PY@tc##1{\textcolor[rgb]{0.73,0.13,0.13}{##1}}}
\@namedef{PY@tok@mb}{\def\PY@tc##1{\textcolor[rgb]{0.40,0.40,0.40}{##1}}}
\@namedef{PY@tok@mf}{\def\PY@tc##1{\textcolor[rgb]{0.40,0.40,0.40}{##1}}}
\@namedef{PY@tok@mh}{\def\PY@tc##1{\textcolor[rgb]{0.40,0.40,0.40}{##1}}}
\@namedef{PY@tok@mi}{\def\PY@tc##1{\textcolor[rgb]{0.40,0.40,0.40}{##1}}}
\@namedef{PY@tok@il}{\def\PY@tc##1{\textcolor[rgb]{0.40,0.40,0.40}{##1}}}
\@namedef{PY@tok@mo}{\def\PY@tc##1{\textcolor[rgb]{0.40,0.40,0.40}{##1}}}
\@namedef{PY@tok@ch}{\let\PY@it=\textit\def\PY@tc##1{\textcolor[rgb]{0.24,0.48,0.48}{##1}}}
\@namedef{PY@tok@cm}{\let\PY@it=\textit\def\PY@tc##1{\textcolor[rgb]{0.24,0.48,0.48}{##1}}}
\@namedef{PY@tok@cpf}{\let\PY@it=\textit\def\PY@tc##1{\textcolor[rgb]{0.24,0.48,0.48}{##1}}}
\@namedef{PY@tok@c1}{\let\PY@it=\textit\def\PY@tc##1{\textcolor[rgb]{0.24,0.48,0.48}{##1}}}
\@namedef{PY@tok@cs}{\let\PY@it=\textit\def\PY@tc##1{\textcolor[rgb]{0.24,0.48,0.48}{##1}}}

\def\PYZbs{\char`\\}
\def\PYZus{\char`\_}
\def\PYZob{\char`\{}
\def\PYZcb{\char`\}}
\def\PYZca{\char`\^}
\def\PYZam{\char`\&}
\def\PYZlt{\char`\<}
\def\PYZgt{\char`\>}
\def\PYZsh{\char`\#}
\def\PYZpc{\char`\%}
\def\PYZdl{\char`\$}
\def\PYZhy{\char`\-}
\def\PYZsq{\char`\'}
\def\PYZdq{\char`\"}
\def\PYZti{\char`\~}
% for compatibility with earlier versions
\def\PYZat{@}
\def\PYZlb{[}
\def\PYZrb{]}
\makeatother


    % For linebreaks inside Verbatim environment from package fancyvrb. 
    \makeatletter
        \newbox\Wrappedcontinuationbox 
        \newbox\Wrappedvisiblespacebox 
        \newcommand*\Wrappedvisiblespace {\textcolor{red}{\textvisiblespace}} 
        \newcommand*\Wrappedcontinuationsymbol {\textcolor{red}{\llap{\tiny$\m@th\hookrightarrow$}}} 
        \newcommand*\Wrappedcontinuationindent {3ex } 
        \newcommand*\Wrappedafterbreak {\kern\Wrappedcontinuationindent\copy\Wrappedcontinuationbox} 
        % Take advantage of the already applied Pygments mark-up to insert 
        % potential linebreaks for TeX processing. 
        %        {, <, #, %, $, ' and ": go to next line. 
        %        _, }, ^, &, >, - and ~: stay at end of broken line. 
        % Use of \textquotesingle for straight quote. 
        \newcommand*\Wrappedbreaksatspecials {% 
            \def\PYGZus{\discretionary{\char`\_}{\Wrappedafterbreak}{\char`\_}}% 
            \def\PYGZob{\discretionary{}{\Wrappedafterbreak\char`\{}{\char`\{}}% 
            \def\PYGZcb{\discretionary{\char`\}}{\Wrappedafterbreak}{\char`\}}}% 
            \def\PYGZca{\discretionary{\char`\^}{\Wrappedafterbreak}{\char`\^}}% 
            \def\PYGZam{\discretionary{\char`\&}{\Wrappedafterbreak}{\char`\&}}% 
            \def\PYGZlt{\discretionary{}{\Wrappedafterbreak\char`\<}{\char`\<}}% 
            \def\PYGZgt{\discretionary{\char`\>}{\Wrappedafterbreak}{\char`\>}}% 
            \def\PYGZsh{\discretionary{}{\Wrappedafterbreak\char`\#}{\char`\#}}% 
            \def\PYGZpc{\discretionary{}{\Wrappedafterbreak\char`\%}{\char`\%}}% 
            \def\PYGZdl{\discretionary{}{\Wrappedafterbreak\char`\$}{\char`\$}}% 
            \def\PYGZhy{\discretionary{\char`\-}{\Wrappedafterbreak}{\char`\-}}% 
            \def\PYGZsq{\discretionary{}{\Wrappedafterbreak\textquotesingle}{\textquotesingle}}% 
            \def\PYGZdq{\discretionary{}{\Wrappedafterbreak\char`\"}{\char`\"}}% 
            \def\PYGZti{\discretionary{\char`\~}{\Wrappedafterbreak}{\char`\~}}% 
        } 
        % Some characters . , ; ? ! / are not pygmentized. 
        % This macro makes them "active" and they will insert potential linebreaks 
        \newcommand*\Wrappedbreaksatpunct {% 
            \lccode`\~`\.\lowercase{\def~}{\discretionary{\hbox{\char`\.}}{\Wrappedafterbreak}{\hbox{\char`\.}}}% 
            \lccode`\~`\,\lowercase{\def~}{\discretionary{\hbox{\char`\,}}{\Wrappedafterbreak}{\hbox{\char`\,}}}% 
            \lccode`\~`\;\lowercase{\def~}{\discretionary{\hbox{\char`\;}}{\Wrappedafterbreak}{\hbox{\char`\;}}}% 
            \lccode`\~`\:\lowercase{\def~}{\discretionary{\hbox{\char`\:}}{\Wrappedafterbreak}{\hbox{\char`\:}}}% 
            \lccode`\~`\?\lowercase{\def~}{\discretionary{\hbox{\char`\?}}{\Wrappedafterbreak}{\hbox{\char`\?}}}% 
            \lccode`\~`\!\lowercase{\def~}{\discretionary{\hbox{\char`\!}}{\Wrappedafterbreak}{\hbox{\char`\!}}}% 
            \lccode`\~`\/\lowercase{\def~}{\discretionary{\hbox{\char`\/}}{\Wrappedafterbreak}{\hbox{\char`\/}}}% 
            \catcode`\.\active
            \catcode`\,\active 
            \catcode`\;\active
            \catcode`\:\active
            \catcode`\?\active
            \catcode`\!\active
            \catcode`\/\active 
            \lccode`\~`\~ 	
        }
    \makeatother

    \let\OriginalVerbatim=\Verbatim
    \makeatletter
    \renewcommand{\Verbatim}[1][1]{%
        %\parskip\z@skip
        \sbox\Wrappedcontinuationbox {\Wrappedcontinuationsymbol}%
        \sbox\Wrappedvisiblespacebox {\FV@SetupFont\Wrappedvisiblespace}%
        \def\FancyVerbFormatLine ##1{\hsize\linewidth
            \vtop{\raggedright\hyphenpenalty\z@\exhyphenpenalty\z@
                \doublehyphendemerits\z@\finalhyphendemerits\z@
                \strut ##1\strut}%
        }%
        % If the linebreak is at a space, the latter will be displayed as visible
        % space at end of first line, and a continuation symbol starts next line.
        % Stretch/shrink are however usually zero for typewriter font.
        \def\FV@Space {%
            \nobreak\hskip\z@ plus\fontdimen3\font minus\fontdimen4\font
            \discretionary{\copy\Wrappedvisiblespacebox}{\Wrappedafterbreak}
            {\kern\fontdimen2\font}%
        }%
        
        % Allow breaks at special characters using \PYG... macros.
        \Wrappedbreaksatspecials
        % Breaks at punctuation characters . , ; ? ! and / need catcode=\active 	
        \OriginalVerbatim[#1,codes*=\Wrappedbreaksatpunct]%
    }
    \makeatother

    % Exact colors from NB
    \definecolor{incolor}{HTML}{303F9F}
    \definecolor{outcolor}{HTML}{D84315}
    \definecolor{cellborder}{HTML}{CFCFCF}
    \definecolor{cellbackground}{HTML}{F7F7F7}
    
    % prompt
    \makeatletter
    \newcommand{\boxspacing}{\kern\kvtcb@left@rule\kern\kvtcb@boxsep}
    \makeatother
    \newcommand{\prompt}[4]{
        {\ttfamily\llap{{\color{#2}[#3]:\hspace{3pt}#4}}\vspace{-\baselineskip}}
    }
    

    
    % Prevent overflowing lines due to hard-to-break entities
    \sloppy 
    % Setup hyperref package
    \hypersetup{
      breaklinks=true,  % so long urls are correctly broken across lines
      colorlinks=true,
      urlcolor=urlcolor,
      linkcolor=linkcolor,
      citecolor=citecolor,
      }
    % Slightly bigger margins than the latex defaults
    
    \geometry{verbose,tmargin=1in,bmargin=1in,lmargin=1in,rmargin=1in}
    
    

\begin{document}
    
    \maketitle
    
    

    
    \begin{tcolorbox}[breakable, size=fbox, boxrule=1pt, pad at break*=1mm,colback=cellbackground, colframe=cellborder]
\prompt{In}{incolor}{32}{\boxspacing}
\begin{Verbatim}[commandchars=\\\{\}]
\PY{k+kn}{import} \PY{n+nn}{numpy} \PY{k}{as} \PY{n+nn}{np}
\PY{k+kn}{import} \PY{n+nn}{matplotlib}\PY{n+nn}{.}\PY{n+nn}{pyplot} \PY{k}{as} \PY{n+nn}{plt}
\PY{k+kn}{from} \PY{n+nn}{astropy}\PY{n+nn}{.}\PY{n+nn}{io} \PY{k+kn}{import} \PY{n}{fits}
\PY{k+kn}{import} \PY{n+nn}{math}
\PY{k+kn}{from} \PY{n+nn}{scipy}\PY{n+nn}{.}\PY{n+nn}{optimize} \PY{k+kn}{import} \PY{n}{curve\PYZus{}fit}
\PY{k+kn}{import} \PY{n+nn}{glob}
\PY{k+kn}{import} \PY{n+nn}{os}
\PY{k+kn}{import} \PY{n+nn}{sys}
\PY{k+kn}{import} \PY{n+nn}{matplotlib}\PY{n+nn}{.}\PY{n+nn}{backends}\PY{n+nn}{.}\PY{n+nn}{backend\PYZus{}pdf}
\PY{k+kn}{import} \PY{n+nn}{matplotlib}\PY{n+nn}{.}\PY{n+nn}{gridspec} \PY{k}{as} \PY{n+nn}{gridspec}
\PY{k+kn}{from} \PY{n+nn}{scipy}\PY{n+nn}{.}\PY{n+nn}{stats} \PY{k+kn}{import} \PY{n}{norm}
\PY{k+kn}{from} \PY{n+nn}{scipy}\PY{n+nn}{.}\PY{n+nn}{stats} \PY{k+kn}{import} \PY{n}{lognorm}
\PY{k+kn}{from} \PY{n+nn}{scipy} \PY{k+kn}{import} \PY{n}{linalg}
\PY{k+kn}{import} \PY{n+nn}{pprint}
\PY{k+kn}{import} \PY{n+nn}{re}
\PY{n}{plt}\PY{o}{.}\PY{n}{rc}\PY{p}{(}\PY{l+s+s1}{\PYZsq{}}\PY{l+s+s1}{font}\PY{l+s+s1}{\PYZsq{}}\PY{p}{,} \PY{n}{family}\PY{o}{=}\PY{l+s+s1}{\PYZsq{}}\PY{l+s+s1}{serif}\PY{l+s+s1}{\PYZsq{}}\PY{p}{)}
\PY{k+kn}{from} \PY{n+nn}{matplotlib} \PY{k+kn}{import} \PY{n}{rc}
\PY{k+kn}{import} \PY{n+nn}{pandas} \PY{k}{as} \PY{n+nn}{pd}
\PY{k+kn}{from} \PY{n+nn}{astropy} \PY{k+kn}{import} \PY{n}{constants} \PY{k}{as} \PY{n}{const}
\PY{k+kn}{from} \PY{n+nn}{astropy} \PY{k+kn}{import} \PY{n}{units} \PY{k}{as} \PY{n}{units}
\PY{k+kn}{from} \PY{n+nn}{IPython}\PY{n+nn}{.}\PY{n+nn}{display} \PY{k+kn}{import} \PY{n}{Markdown} \PY{k}{as} \PY{n}{md}
\PY{k+kn}{import} \PY{n+nn}{sympy}
\PY{k+kn}{import} \PY{n+nn}{pandas}
\PY{k+kn}{from} \PY{n+nn}{sympy} \PY{k+kn}{import} \PY{n}{integrate}\PY{p}{,} \PY{n}{diff}\PY{p}{,} \PY{n}{sqrt}\PY{p}{,} \PY{n}{cos}\PY{p}{,} \PY{n}{sin}\PY{p}{,} \PY{n}{pi}\PY{p}{,} \PY{n}{exp}\PY{p}{,} \PY{n}{log}
\PY{k+kn}{from} \PY{n+nn}{sympy}\PY{n+nn}{.}\PY{n+nn}{abc} \PY{k+kn}{import} \PY{o}{*} 
\PY{n}{i} \PY{o}{=} \PY{n}{sqrt}\PY{p}{(}\PY{o}{\PYZhy{}}\PY{l+m+mi}{1}\PY{p}{)}
\PY{k+kn}{import} \PY{n+nn}{numpy} \PY{k}{as} \PY{n+nn}{np}
\PY{k+kn}{import} \PY{n+nn}{sympy}\PY{n+nn}{.}\PY{n+nn}{printing} \PY{k}{as} \PY{n+nn}{printing}
\PY{n}{latp} \PY{o}{=} \PY{n}{printing}\PY{o}{.}\PY{n}{latex}
\PY{n}{hbar} \PY{o}{=} \PY{n}{sympy}\PY{o}{.}\PY{n}{symbols}\PY{p}{(}\PY{l+s+s2}{\PYZdq{}}\PY{l+s+s2}{hbar}\PY{l+s+s2}{\PYZdq{}}\PY{p}{,} \PY{n}{real}\PY{o}{=}\PY{k+kc}{True}\PY{p}{)}
\PY{k+kn}{import} \PY{n+nn}{mpmath} 
\PY{k+kn}{import} \PY{n+nn}{plotly}\PY{n+nn}{.}\PY{n+nn}{express} \PY{k}{as} \PY{n+nn}{px}
\PY{k+kn}{import} \PY{n+nn}{plotly}\PY{n+nn}{.}\PY{n+nn}{graph\PYZus{}objs} \PY{k}{as} \PY{n+nn}{go}
\PY{n}{hbar}
\end{Verbatim}
\end{tcolorbox}
 
            
\prompt{Out}{outcolor}{32}{}
    
    $\displaystyle \hbar$

    

    We now follow the directions in the background for the following
derivations. \begin{align}
&\text{Grating Equation}\quad\ m\lambda\frac{1}{a} = \sin\alpha\pm\sin\beta\\
&\text{Since we use a reflective grating}\quad\ m\lambda\frac{1}{a} = \sin\alpha+\sin\beta\\ 
&\text{Taking the partial derivative with respect to }\lambda\quad\  
\frac{m}{a} = \frac{\partial \beta}{\partial \lambda}\cos\beta\\ 
&\text{Rearranging}\quad\  \frac{m}{a\cos\beta} = \frac{\partial \beta}{\partial \lambda}\\
&\text{Rearranging the grating Equation}\quad\ \lambda = \frac{a}{m}(\sin\alpha+\sin\beta)\\
&\text{Multiplying the above two lines}\quad\  \frac{m}{a\cos\beta}\frac{a}{m}(\sin\alpha+\sin\beta) = \lambda\frac{\partial \beta}{\partial \lambda}\\
&\text{Simplifying the left side}\quad\  \frac{\sin\alpha+\sin\beta}{\cos\beta } = \lambda\frac{\partial \beta}{\partial \lambda}\\
&\text{Since }\alpha -2\theta=\beta \quad\ \frac{\sin(\beta+2\theta)+\sin\beta}{\cos\beta } = \lambda\frac{\partial \beta}{\partial \lambda}\\
&\text{Simplifying and substituting tangent}\quad\ \frac{\sin(\beta+2\theta)}{\cos\beta } + \tan( \beta) = \lambda\frac{\partial \beta}{\partial \lambda}\\
&\text{Since }\sin(a+b)= \sin(a)\cos(b)+\cos(a)\sin(b)\quad\ \frac{\sin(2\theta)\cos(\beta)+\cos(2\theta)\sin(\beta)}{\cos\beta } + \tan( \beta) = \lambda\frac{\partial \beta}{\partial \lambda}\\
&\text{Simplifying and factoring}\quad\  
\sin(2\theta) + \tan( \beta)(1+\cos(2\theta)) = \lambda\frac{\partial \beta}{\partial \lambda}\\ 
&\text{Since }2\cos^2(\theta)-1=\cos(2\theta)\quad\ 
\sin(2\theta) + 2\tan(\beta)\cos^2(\theta) = \lambda\frac{\partial \beta}{\partial \lambda}\\ 
&\blacksquare\nonumber
\end{align}

    \begin{align}
&\text{Grating Equation}\quad\ \frac{m\lambda}{a} = \sin(\alpha)+\sin(\beta)\\
&\text{Since }2\theta=\alpha-\beta\quad\ \frac{m\lambda}{a} = \sin(\alpha)+\sin(\alpha-2\theta)\\ 
&\text{Since }2\theta=\alpha-\beta\quad\ \frac{m\lambda}{a} = \sin(\alpha)+\sin(\alpha-2\theta)\\ 
&\text{Sine addition formula}\quad\ \frac{m\lambda}{a} = \sin\alpha(1+\cos(2\theta))-\cos(\alpha)\sin(2\theta)\\
&\text{Trigonometric identity}\quad\ \frac{m\lambda}{a} = 2\sin\alpha\cos^2(\theta))-\cos(\alpha)\sin(2\theta)\\
&\text{Sine addition formula}\quad\ \frac{m\lambda}{a} =  2\sin\alpha\cos^2(\theta))-2\cos(\theta)\cos(\alpha)\sin(\theta)\\
&\text{Rearranging}\quad\ \frac{m\lambda}{2a\cos\theta} =  \sin\alpha\cos(\theta))-\cos(\alpha)\sin(\theta)\\ 
&\text{Sine difference formula}\quad\ \frac{m\lambda}{2a\cos\theta} =
\sin(\alpha-\theta)\\
&\text{Solve for }\alpha\quad\ \arcsin\left(\frac{m\lambda}{2a\cos\theta}\right)+\theta =
\alpha
\end{align}

    \begin{align}
    &\text{We know}\quad\ D=i-A+\arcsin\left(n\sin(A-\arcsin\left(\frac{\sin   i}{n}\right)\right)
\end{align}

    The desired result can be derived by differentiating both sides with
respect to \(i\) and then solving for \(n.\) Since at the desired \(n\),
\(\frac{dD}{di}=0\) one can use trigonometric identities to
algebraically rearrange and solve for \(n\).

    \begin{tcolorbox}[breakable, size=fbox, boxrule=1pt, pad at break*=1mm,colback=cellbackground, colframe=cellborder]
\prompt{In}{incolor}{135}{\boxspacing}
\begin{Verbatim}[commandchars=\\\{\}]
\PY{n}{data} \PY{o}{=} \PY{n}{pd}\PY{o}{.}\PY{n}{read\PYZus{}csv}\PY{p}{(}\PY{l+s+s2}{\PYZdq{}}\PY{l+s+s2}{lab4 \PYZhy{} The Cooler First Order.csv}\PY{l+s+s2}{\PYZdq{}}\PY{p}{)}
\PY{n}{data}\PY{p}{,} \PY{n}{end\PYZus{}of\PYZus{}data} \PY{o}{=} \PY{n}{data}\PY{o}{.}\PY{n}{iloc}\PY{p}{[}\PY{p}{:}\PY{l+m+mi}{3}\PY{p}{,}\PY{p}{:}\PY{p}{]}\PY{p}{,} \PY{n}{data}\PY{o}{.}\PY{n}{iloc}\PY{p}{[}\PY{l+m+mi}{3}\PY{p}{:}\PY{p}{,}\PY{p}{:}\PY{p}{]}
\PY{n}{end\PYZus{}of\PYZus{}data} \PY{o}{=} \PY{p}{[}\PY{n}{x} \PY{k}{for} \PY{n}{x} \PY{o+ow}{in} \PY{n}{end\PYZus{}of\PYZus{}data}\PY{o}{.}\PY{n}{iloc}\PY{p}{[}\PY{p}{:}\PY{p}{,}\PY{l+m+mi}{0}\PY{p}{]} \PY{k}{if} \PY{n+nb}{len}\PY{p}{(}\PY{n+nb}{str}\PY{p}{(}\PY{n}{x}\PY{p}{)}\PY{p}{)}\PY{o}{\PYZgt{}}\PY{l+m+mi}{0}
              \PY{k}{if} \PY{n+nb}{type}\PY{p}{(}\PY{n}{x}\PY{p}{)} \PY{o+ow}{is} \PY{n+nb}{str}\PY{p}{]}
\PY{n}{data}
\end{Verbatim}
\end{tcolorbox}

            \begin{tcolorbox}[breakable, size=fbox, boxrule=.5pt, pad at break*=1mm, opacityfill=0]
\prompt{Out}{outcolor}{135}{\boxspacing}
\begin{Verbatim}[commandchars=\\\{\}]
  Slit Width Microns Separation Between Each Line (mm) on Screen  \textbackslash{}
0          15 \textbackslash{}pm .5                              5 mm \textbackslash{}pm .5 mm
1        19.5 \textbackslash{}pm .5                             5 cm \textbackslash{}pm 0.5 mm
2          8 \textbackslash{}pm 0.5                                Not Recorded

            Rotational Stage Reading When Blazed  \textbackslash{}
0      97 degrees 55 arcminutes \textbackslash{}pm 5 arcminutes
1      98 degrees 10 arcminutes \textbackslash{}pm 5 arcminutes
2  97 degrees and 50 arcminutes \textbackslash{}pm 5 arcminutes

  Rotational stage reading when the diffraction grating is normal to the beam  \textbackslash{}
0          95 degrees 35 arcminutes \textbackslash{}pm 5 arcminutes
1          95 degrees 25 arcminutes \textbackslash{}pm 5 arcminutes
2       96 degrees and 5 arcminutes \textbackslash{}pm 5 arcminutes

  Brighter Further Line Separation From Incoming Beam  \textbackslash{}
0                                       74 \textbackslash{}pm .7 mm
1                                 75.5 mm \textbackslash{}pm 0.7 mm
2                                 86.0 mm \textbackslash{}pm 0.7 mm

  Dimmer Closer Line Separation From Incoming Beam
0                                     64 \textbackslash{}pm .7 mm
1                               70.0 mm \textbackslash{}pm 0.7 mm
2                               76.5 mm \textbackslash{}pm 0.7 mm
\end{Verbatim}
\end{tcolorbox}
        
    \begin{tcolorbox}[breakable, size=fbox, boxrule=1pt, pad at break*=1mm,colback=cellbackground, colframe=cellborder]
\prompt{In}{incolor}{96}{\boxspacing}
\begin{Verbatim}[commandchars=\\\{\}]
\PY{k}{for} \PY{n}{note} \PY{o+ow}{in} \PY{n}{end\PYZus{}of\PYZus{}data}\PY{p}{:}
    \PY{n+nb}{print}\PY{p}{(}\PY{n}{note}\PY{p}{)}
\end{Verbatim}
\end{tcolorbox}

    \begin{Verbatim}[commandchars=\\\{\}]
distance from diffraction grating to red line on our piece of paper 97.5 \textbackslash{}pm 0.7
mm
1200 lines/mm diffraction grating
    \end{Verbatim}

    \begin{tcolorbox}[breakable, size=fbox, boxrule=1pt, pad at break*=1mm,colback=cellbackground, colframe=cellborder]
\prompt{In}{incolor}{199}{\boxspacing}
\begin{Verbatim}[commandchars=\\\{\}]
\PY{k}{def} \PY{n+nf}{convert\PYZus{}data}\PY{p}{(}\PY{n}{element}\PY{p}{)}\PY{p}{:}
    \PY{k}{try}\PY{p}{:}
        \PY{n}{expected\PYZus{}val}\PY{p}{,} \PY{n}{uncertainty} \PY{o}{=} \PY{p}{[}\PY{n+nb}{float}\PY{p}{(}\PY{n}{x}\PY{p}{)} \PY{k}{for} \PY{n}{x} 
            \PY{o+ow}{in} \PY{n}{element}\PY{o}{.}\PY{n}{split}\PY{p}{(}\PY{l+s+s2}{\PYZdq{}}\PY{l+s+s2}{ }\PY{l+s+s2}{\PYZdq{}}\PY{p}{)}
            \PY{k}{if} \PY{n+nb}{len}\PY{p}{(}\PY{n}{x}\PY{p}{)}\PY{o}{\PYZgt{}}\PY{l+m+mi}{0} \PY{o+ow}{and} \PY{p}{(}\PY{n}{x}\PY{p}{[}\PY{o}{\PYZhy{}}\PY{l+m+mi}{1}\PY{p}{]}\PY{o}{.}\PY{n}{isnumeric}\PY{p}{(}\PY{p}{)} \PY{o+ow}{or} \PY{n}{x}\PY{p}{[}\PY{l+m+mi}{0}\PY{p}{]}\PY{o}{.}\PY{n}{isnumeric}\PY{p}{(}\PY{p}{)}\PY{p}{)}\PY{p}{]}
        \PY{k}{return} \PY{n}{expected\PYZus{}val}\PY{p}{,} \PY{n}{uncertainty}
    \PY{k}{except}\PY{p}{:} 
        \PY{k}{try}\PY{p}{:} 
            \PY{n}{expected\PYZus{}val}\PY{p}{,} \PY{n}{arcmin}\PY{p}{,} \PY{n}{uncertainty} \PY{o}{=} \PY{p}{[}\PY{n+nb}{float}\PY{p}{(}\PY{n}{x}\PY{p}{)} \PY{k}{for} \PY{n}{x} 
            \PY{o+ow}{in} \PY{n}{element}\PY{o}{.}\PY{n}{split}\PY{p}{(}\PY{l+s+s2}{\PYZdq{}}\PY{l+s+s2}{ }\PY{l+s+s2}{\PYZdq{}}\PY{p}{)}
            \PY{k}{if} \PY{n}{x}\PY{p}{[}\PY{o}{\PYZhy{}}\PY{l+m+mi}{1}\PY{p}{]}\PY{o}{.}\PY{n}{isnumeric}\PY{p}{(}\PY{p}{)} \PY{o+ow}{or} \PY{n}{x}\PY{p}{[}\PY{l+m+mi}{0}\PY{p}{]}\PY{o}{.}\PY{n}{isnumeric}\PY{p}{(}\PY{p}{)}\PY{p}{]}
            \PY{n}{expected\PYZus{}val} \PY{o}{=} \PY{n}{expected\PYZus{}val} \PY{o}{+} \PY{n}{arcmin} \PY{o}{/} \PY{l+m+mi}{60}
            \PY{k}{return} \PY{n}{expected\PYZus{}val}\PY{p}{,} \PY{n}{uncertainty} \PY{o}{/} \PY{l+m+mi}{60}
        \PY{k}{except}\PY{p}{:}
            \PY{k}{return} \PY{l+s+s2}{\PYZdq{}}\PY{l+s+s2}{NA}\PY{l+s+s2}{\PYZdq{}}
\PY{k}{def} \PY{n+nf}{arr\PYZus{}to\PYZus{}dict}\PY{p}{(}\PY{n}{arr}\PY{p}{)}\PY{p}{:}
    \PY{n}{out} \PY{o}{=} \PY{p}{\PYZob{}}\PY{p}{\PYZcb{}}
    \PY{k}{for} \PY{n}{pair} \PY{o+ow}{in} \PY{n}{arr}\PY{p}{:}
        \PY{n}{out}\PY{p}{[}\PY{n}{pair}\PY{p}{[}\PY{l+m+mi}{0}\PY{p}{]}\PY{p}{]} \PY{o}{=} \PY{n}{pair}\PY{p}{[}\PY{l+m+mi}{1}\PY{p}{]}
    \PY{k}{return} \PY{n}{out}
\PY{k}{def} \PY{n+nf}{convert\PYZus{}col}\PY{p}{(}\PY{n}{df}\PY{p}{,} \PY{n}{col}\PY{p}{)}\PY{p}{:}
    \PY{k}{return} \PY{n}{arr\PYZus{}to\PYZus{}dict}\PY{p}{(}\PY{p}{[}\PY{n}{convert\PYZus{}data}\PY{p}{(}\PY{n}{x}\PY{p}{)} \PY{k}{for} \PY{n}{x} \PY{o+ow}{in} \PY{n}{data}\PY{p}{[}\PY{n}{col}\PY{p}{]}\PY{p}{]}\PY{p}{)}
\PY{n}{slit\PYZus{}widths} \PY{o}{=} \PY{n}{convert\PYZus{}col}\PY{p}{(}\PY{n}{data}\PY{p}{,} \PY{l+s+s1}{\PYZsq{}}\PY{l+s+s1}{Slit Width Microns}\PY{l+s+s1}{\PYZsq{}}\PY{p}{)}
\PY{n}{slit\PYZus{}widths}
\end{Verbatim}
\end{tcolorbox}

            \begin{tcolorbox}[breakable, size=fbox, boxrule=.5pt, pad at break*=1mm, opacityfill=0]
\prompt{Out}{outcolor}{199}{\boxspacing}
\begin{Verbatim}[commandchars=\\\{\}]
\{15.0: 0.5, 19.5: 0.5, 8.0: 0.5\}
\end{Verbatim}
\end{tcolorbox}
        
    We now perform an average weighted linearly by the slit width, since it
should be proportional to the photon flux on the screen.

    \begin{tcolorbox}[breakable, size=fbox, boxrule=1pt, pad at break*=1mm,colback=cellbackground, colframe=cellborder]
\prompt{In}{incolor}{200}{\boxspacing}
\begin{Verbatim}[commandchars=\\\{\}]
\PY{n}{weights} \PY{o}{=} \PY{p}{[}\PY{n}{x} \PY{k}{for} \PY{n}{x} \PY{o+ow}{in} \PY{n}{slit\PYZus{}widths}\PY{o}{.}\PY{n}{keys}\PY{p}{(}\PY{p}{)}\PY{p}{]}
\PY{n}{n} \PY{o}{=} \PY{n+nb}{len}\PY{p}{(}\PY{n}{weights}\PY{p}{)}
\end{Verbatim}
\end{tcolorbox}

    \begin{tcolorbox}[breakable, size=fbox, boxrule=1pt, pad at break*=1mm,colback=cellbackground, colframe=cellborder]
\prompt{In}{incolor}{201}{\boxspacing}
\begin{Verbatim}[commandchars=\\\{\}]
\PY{k}{def} \PY{n+nf}{weighted\PYZus{}average}\PY{p}{(}\PY{n}{data}\PY{p}{,} \PY{n}{col}\PY{p}{,} \PY{n}{weights}\PY{p}{,} \PY{n}{coeff}\PY{o}{=}\PY{l+m+mi}{1}\PY{p}{)}\PY{p}{:}
    \PY{n}{column} \PY{o}{=} \PY{n}{convert\PYZus{}col}\PY{p}{(}\PY{n}{data}\PY{p}{,} \PY{n}{col}\PY{p}{)}
    \PY{n}{keys} \PY{o}{=} \PY{n}{coeff} \PY{o}{*} \PY{n}{np}\PY{o}{.}\PY{n}{array}\PY{p}{(}\PY{p}{[}\PY{n}{x} \PY{k}{for} \PY{n}{x} \PY{o+ow}{in} \PY{n}{column}\PY{o}{.}\PY{n}{keys}\PY{p}{(}\PY{p}{)}\PY{p}{]}\PY{p}{)}
    \PY{n}{unc} \PY{o}{=} \PY{n}{coeff} \PY{o}{*} \PY{p}{[}\PY{n}{x} \PY{k}{for} \PY{n}{x} \PY{o+ow}{in} \PY{n}{column}\PY{o}{.}\PY{n}{values}\PY{p}{(}\PY{p}{)}\PY{p}{]}\PY{p}{[}\PY{l+m+mi}{0}\PY{p}{]} \PY{o}{/} \PY{n}{np}\PY{o}{.}\PY{n}{sqrt}\PY{p}{(}
            \PY{n+nb}{len}\PY{p}{(}\PY{n}{weights}\PY{p}{)}\PY{p}{)}
    \PY{k}{return} \PY{p}{\PYZob{}} \PY{n}{np}\PY{o}{.}\PY{n}{sum}\PY{p}{(}\PY{n}{np}\PY{o}{.}\PY{n}{array}\PY{p}{(}\PY{p}{[}\PY{n}{weights}\PY{p}{[}\PY{n}{col}\PY{p}{]} \PY{o}{*} \PY{n}{keys}\PY{p}{[}\PY{n}{col}\PY{p}{]} \PY{k}{for} 
            \PY{n}{col} \PY{o+ow}{in} \PY{n+nb}{range}\PY{p}{(}\PY{n}{data}\PY{o}{.}\PY{n}{shape}\PY{p}{[}\PY{l+m+mi}{0}\PY{p}{]}\PY{p}{)}\PY{p}{]}\PY{p}{)}\PY{p}{)} \PY{o}{/} \PY{n}{np}\PY{o}{.}\PY{n}{sum}\PY{p}{(}\PY{n}{weights}\PY{p}{)} \PY{p}{:} \PY{n}{unc} \PY{p}{\PYZcb{}}
\PY{n}{average\PYZus{}angle\PYZus{}blazed} \PY{o}{=} \PY{n}{weighted\PYZus{}average}\PY{p}{(}\PY{n}{data}\PY{p}{,} 
        \PY{l+s+s1}{\PYZsq{}}\PY{l+s+s1}{Rotational Stage Reading When Blazed}\PY{l+s+s1}{\PYZsq{}}\PY{p}{,} \PY{n}{weights}\PY{p}{,} \PY{n}{np}\PY{o}{.}\PY{n}{pi}\PY{o}{/}\PY{l+m+mi}{180}\PY{p}{)}
\PY{n}{average\PYZus{}angle\PYZus{}blazed}
\end{Verbatim}
\end{tcolorbox}

            \begin{tcolorbox}[breakable, size=fbox, boxrule=.5pt, pad at break*=1mm, opacityfill=0]
\prompt{Out}{outcolor}{201}{\boxspacing}
\begin{Verbatim}[commandchars=\\\{\}]
\{1.7106964440920107: 0.000839721927886212\}
\end{Verbatim}
\end{tcolorbox}
        
    \begin{tcolorbox}[breakable, size=fbox, boxrule=1pt, pad at break*=1mm,colback=cellbackground, colframe=cellborder]
\prompt{In}{incolor}{202}{\boxspacing}
\begin{Verbatim}[commandchars=\\\{\}]
\PY{n}{col} \PY{o}{=} \PY{l+s+s1}{\PYZsq{}}\PY{l+s+s1}{Rotational stage reading when the diffraction grating is normal to the beam}\PY{l+s+s1}{\PYZsq{}}
\PY{n}{average\PYZus{}angle\PYZus{}zeroed} \PY{o}{=} \PY{n}{weighted\PYZus{}average}\PY{p}{(}\PY{n}{data}\PY{p}{,} \PY{n}{col}\PY{p}{,} \PY{n}{weights}\PY{p}{,} \PY{n}{np}\PY{o}{.}\PY{n}{pi}\PY{o}{/}\PY{l+m+mi}{180}\PY{p}{)}
\PY{n}{average\PYZus{}angle\PYZus{}zeroed} 
\end{Verbatim}
\end{tcolorbox}

            \begin{tcolorbox}[breakable, size=fbox, boxrule=.5pt, pad at break*=1mm, opacityfill=0]
\prompt{Out}{outcolor}{202}{\boxspacing}
\begin{Verbatim}[commandchars=\\\{\}]
\{1.668551875977677: 0.000839721927886212\}
\end{Verbatim}
\end{tcolorbox}
        
    \begin{tcolorbox}[breakable, size=fbox, boxrule=1pt, pad at break*=1mm,colback=cellbackground, colframe=cellborder]
\prompt{In}{incolor}{203}{\boxspacing}
\begin{Verbatim}[commandchars=\\\{\}]
\PY{n}{brighter\PYZus{}further} \PY{o}{=} \PY{n}{weighted\PYZus{}average}\PY{p}{(}\PY{n}{data}\PY{p}{,} 
    \PY{l+s+s1}{\PYZsq{}}\PY{l+s+s1}{Brighter Further Line Separation From Incoming Beam}\PY{l+s+s1}{\PYZsq{}}\PY{p}{,} \PY{n}{weights}\PY{p}{)}
\PY{n}{brighter\PYZus{}further}
\end{Verbatim}
\end{tcolorbox}

            \begin{tcolorbox}[breakable, size=fbox, boxrule=.5pt, pad at break*=1mm, opacityfill=0]
\prompt{Out}{outcolor}{203}{\boxspacing}
\begin{Verbatim}[commandchars=\\\{\}]
\{76.94705882352942: 0.40414518843273806\}
\end{Verbatim}
\end{tcolorbox}
        
    \begin{tcolorbox}[breakable, size=fbox, boxrule=1pt, pad at break*=1mm,colback=cellbackground, colframe=cellborder]
\prompt{In}{incolor}{204}{\boxspacing}
\begin{Verbatim}[commandchars=\\\{\}]
\PY{n}{dimmer\PYZus{}closer} \PY{o}{=} \PY{n}{weighted\PYZus{}average}\PY{p}{(}\PY{n}{data}\PY{p}{,} 
    \PY{l+s+s1}{\PYZsq{}}\PY{l+s+s1}{Dimmer Closer Line Separation From Incoming Beam}\PY{l+s+s1}{\PYZsq{}}\PY{p}{,} \PY{n}{weights}\PY{p}{)}
\PY{n}{dimmer\PYZus{}closer}
\end{Verbatim}
\end{tcolorbox}

            \begin{tcolorbox}[breakable, size=fbox, boxrule=.5pt, pad at break*=1mm, opacityfill=0]
\prompt{Out}{outcolor}{204}{\boxspacing}
\begin{Verbatim}[commandchars=\\\{\}]
\{69.10588235294118: 0.40414518843273806\}
\end{Verbatim}
\end{tcolorbox}
        
    I have written a publicly available error propagation Python script on
Github, which I make use of here. The purpose of this file is to have a
set of functions that can calculate error propagation. All inputs will
be dictionaries of the form

dict = \{value1: uncertainty1, value2: uncertainty2, \ldots\}

All outputs will be lists of the form (value, uncertainty). All
uncertanties should be listed as nonnegative real numbers. For the
general() function to calculate the exact uncertainty, the syntax for
defining functions used here:

https://www.askpython.com/python/examples/derivatives-in-python-sympy

should be used. The purpose of symbol\_list list is to have up to 26
variable names to take partial differentials in general() All functions
used in the general function must use the first n letters of the
alphabet as their variables

    \begin{tcolorbox}[breakable, size=fbox, boxrule=1pt, pad at break*=1mm,colback=cellbackground, colframe=cellborder]
\prompt{In}{incolor}{205}{\boxspacing}
\begin{Verbatim}[commandchars=\\\{\}]
\PY{n}{a}\PY{p}{,} \PY{n}{b}\PY{p}{,} \PY{n}{c}\PY{p}{,} \PY{n}{d}\PY{p}{,} \PY{n}{e}\PY{p}{,} \PY{n}{f}\PY{p}{,} \PY{n}{g}\PY{p}{,} \PY{n}{h}\PY{p}{,} \PY{n}{i}\PY{p}{,} \PY{n}{j}\PY{p}{,} \PY{n}{k}\PY{p}{,} \PY{n}{l}\PY{p}{,} \PY{n}{m} \PY{o}{=} \PY{n}{symbols}\PY{p}{(}\PY{l+s+s1}{\PYZsq{}}\PY{l+s+s1}{a b c d e f g h i j k l m}\PY{l+s+s1}{\PYZsq{}}\PY{p}{)}
\PY{n}{n}\PY{p}{,} \PY{n}{o}\PY{p}{,} \PY{n}{p}\PY{p}{,} \PY{n}{q}\PY{p}{,} \PY{n}{r}\PY{p}{,} \PY{n}{s}\PY{p}{,} \PY{n}{t}\PY{p}{,} \PY{n}{u}\PY{p}{,} \PY{n}{v}\PY{p}{,} \PY{n}{w}\PY{p}{,} \PY{n}{x}\PY{p}{,} \PY{n}{y}\PY{p}{,} \PY{n}{z} \PY{o}{=} \PY{n}{symbols}\PY{p}{(}\PY{l+s+s1}{\PYZsq{}}\PY{l+s+s1}{n o p q r s t u v w x y z}\PY{l+s+s1}{\PYZsq{}}\PY{p}{)}
\PY{n}{symbol\PYZus{}list} \PY{o}{=} \PY{p}{(}\PY{n}{a}\PY{p}{,} \PY{n}{b}\PY{p}{,} \PY{n}{c}\PY{p}{,} \PY{n}{d}\PY{p}{,} \PY{n}{e}\PY{p}{,} \PY{n}{f}\PY{p}{,} \PY{n}{g}\PY{p}{,} \PY{n}{h}\PY{p}{,} \PY{n}{i}\PY{p}{,} \PY{n}{j}\PY{p}{,} \PY{n}{k}\PY{p}{,} \PY{n}{l}\PY{p}{,} \PY{n}{m}\PY{p}{,} \PY{n}{n}\PY{p}{,} \PY{n}{o}\PY{p}{,} \PY{n}{p}\PY{p}{,} \PY{n}{q}\PY{p}{,} \PY{n}{r}\PY{p}{,} \PY{n}{s}\PY{p}{,} \PY{n}{t}\PY{p}{,} \PY{n}{u}\PY{p}{,} \PY{n}{v}\PY{p}{,}
\PY{n}{w}\PY{p}{,} \PY{n}{x}\PY{p}{,} \PY{n}{y}\PY{p}{,} \PY{n}{z}\PY{p}{)}

\PY{k}{def} \PY{n+nf}{general} \PY{p}{(}\PY{n+nb}{dict}\PY{p}{,} \PY{n}{fun}\PY{p}{)}\PY{p}{:}
    \PY{c+c1}{\PYZsh{}can only take up to 26 inputs}
    \PY{c+c1}{\PYZsh{}its output is of the form (exact uncertainty, maximum uncertainty)}
    \PY{n}{sum\PYZus{}squares} \PY{o}{=} \PY{l+m+mi}{0}

    \PY{k}{try}\PY{p}{:}
        \PY{n}{keys} \PY{o}{=} \PY{n}{np}\PY{o}{.}\PY{n}{array}\PY{p}{(}\PY{n+nb}{list}\PY{p}{(}\PY{n+nb}{dict}\PY{o}{.}\PY{n}{keys}\PY{p}{(}\PY{p}{)}\PY{p}{)}\PY{p}{)}
        \PY{n}{values} \PY{o}{=} \PY{n}{np}\PY{o}{.}\PY{n}{array}\PY{p}{(}\PY{n+nb}{list}\PY{p}{(}\PY{n+nb}{dict}\PY{o}{.}\PY{n}{values}\PY{p}{(}\PY{p}{)}\PY{p}{)}\PY{p}{)}
        \PY{k}{for} \PY{n}{index} \PY{o+ow}{in} \PY{n+nb}{range}\PY{p}{(}\PY{n+nb}{len}\PY{p}{(}\PY{n}{keys}\PY{p}{)}\PY{p}{)}\PY{p}{:}
            \PY{n}{diff\PYZus{}f} \PY{o}{=} \PY{n}{fun}\PY{o}{.}\PY{n}{diff}\PY{p}{(}\PY{n}{symbol\PYZus{}list}\PY{p}{[}\PY{n}{index}\PY{p}{]}\PY{p}{)}
            \PY{n}{lam\PYZus{}f} \PY{o}{=} \PY{n}{lambdify}\PY{p}{(}\PY{n}{symbol\PYZus{}list}\PY{p}{[}\PY{l+m+mi}{0}\PY{p}{:}\PY{n+nb}{len}\PY{p}{(}\PY{n}{keys}\PY{p}{)}\PY{p}{]}\PY{p}{,} \PY{n}{diff\PYZus{}f}\PY{p}{)}
            \PY{n}{sum\PYZus{}squares} \PY{o}{+}\PY{o}{=} \PY{p}{(}\PY{n}{lam\PYZus{}f}\PY{p}{(}\PY{o}{*}\PY{n}{keys}\PY{p}{)} \PY{o}{*} \PY{n}{values}\PY{p}{[}\PY{n}{index}\PY{p}{]}\PY{p}{)} \PY{o}{*}\PY{o}{*} \PY{l+m+mi}{2}
        \PY{n}{lam\PYZus{}og} \PY{o}{=} \PY{n}{lambdify}\PY{p}{(}\PY{n}{symbol\PYZus{}list}\PY{p}{[}\PY{l+m+mi}{0}\PY{p}{:}\PY{n+nb}{len}\PY{p}{(}\PY{n}{keys}\PY{p}{)}\PY{p}{]}\PY{p}{,} \PY{n}{fun}\PY{p}{)}
        \PY{n}{best\PYZus{}est} \PY{o}{=} \PY{n}{lam\PYZus{}og}\PY{p}{(}\PY{o}{*}\PY{n}{keys}\PY{p}{)}
    \PY{k}{except}\PY{p}{:}
        \PY{n+nb}{print}\PY{p}{(}\PY{l+s+s2}{\PYZdq{}}\PY{l+s+s2}{Singular uncertainty entered}\PY{l+s+s2}{\PYZdq{}}\PY{p}{)}
        \PY{n}{keys} \PY{o}{=} \PY{n+nb}{list}\PY{p}{(}\PY{n+nb}{dict}\PY{o}{.}\PY{n}{keys}\PY{p}{(}\PY{p}{)}\PY{p}{)}
        \PY{n}{keys} \PY{o}{=} \PY{n+nb}{float}\PY{p}{(}\PY{n}{keys}\PY{p}{[}\PY{l+m+mi}{0}\PY{p}{]}\PY{p}{)}
        \PY{n}{values} \PY{o}{=} \PY{n+nb}{list}\PY{p}{(}\PY{n+nb}{dict}\PY{o}{.}\PY{n}{values}\PY{p}{(}\PY{p}{)}\PY{p}{)}
        \PY{n}{values} \PY{o}{=} \PY{n+nb}{float}\PY{p}{(}\PY{n}{values}\PY{p}{[}\PY{l+m+mi}{0}\PY{p}{]}\PY{p}{)}
        \PY{n}{diff\PYZus{}f} \PY{o}{=} \PY{n}{fun}\PY{o}{.}\PY{n}{diff}\PY{p}{(}\PY{n}{symbol\PYZus{}list}\PY{p}{[}\PY{l+m+mi}{0}\PY{p}{]}\PY{p}{)}
        \PY{n}{lam\PYZus{}f} \PY{o}{=} \PY{n}{lambdify}\PY{p}{(}\PY{n}{symbol\PYZus{}list}\PY{p}{[}\PY{l+m+mi}{0}\PY{p}{]}\PY{p}{,} \PY{n}{diff\PYZus{}f}\PY{p}{)}
        \PY{n}{sum\PYZus{}squares} \PY{o}{+}\PY{o}{=} \PY{p}{(}\PY{n}{lam\PYZus{}f}\PY{p}{(}\PY{n}{keys}\PY{p}{)} \PY{o}{*} \PY{n}{values}\PY{p}{)} \PY{o}{*}\PY{o}{*} \PY{l+m+mi}{2}
        \PY{n}{lam\PYZus{}og} \PY{o}{=} \PY{n}{lambdify}\PY{p}{(}\PY{n}{symbol\PYZus{}list}\PY{p}{[}\PY{l+m+mi}{0}\PY{p}{]}\PY{p}{,} \PY{n}{fun}\PY{p}{)}
        \PY{n}{lam\PYZus{}og}\PY{p}{(}\PY{n}{keys}\PY{p}{)}

    \PY{n}{uncertainty} \PY{o}{=} \PY{n}{math}\PY{o}{.}\PY{n}{sqrt}\PY{p}{(}\PY{n}{sum\PYZus{}squares}\PY{p}{)}

    \PY{k}{return} \PY{p}{(}\PY{n}{best\PYZus{}est}\PY{p}{,} \PY{n}{uncertainty}\PY{p}{)}
\PY{c+c1}{\PYZsh{}a = x, y = b, c = theta}
\PY{n}{fun} \PY{o}{=} \PY{p}{(}\PY{n}{a} \PY{o}{+} \PY{l+m+mi}{2}\PY{p}{)} \PY{o}{/} \PY{p}{(}\PY{n}{a} \PY{o}{+} \PY{p}{(}\PY{n}{b} \PY{o}{*} \PY{n}{cos}\PY{p}{(}\PY{l+m+mi}{4} \PY{o}{*} \PY{n}{c} \PY{o}{*} \PY{n}{np}\PY{o}{.}\PY{n}{pi} \PY{o}{/} \PY{l+m+mi}{180}\PY{p}{)}\PY{p}{)}\PY{p}{)}
\PY{n+nb}{dict} \PY{o}{=} \PY{p}{\PYZob{}}\PY{l+m+mi}{10}\PY{p}{:}\PY{l+m+mi}{2}\PY{p}{,} \PY{l+m+mi}{7}\PY{p}{:}\PY{l+m+mi}{1}\PY{p}{,} \PY{l+m+mi}{40}\PY{p}{:}\PY{l+m+mi}{3}\PY{p}{\PYZcb{}}
\PY{n+nb}{print}\PY{p}{(}\PY{n}{general}\PY{p}{(}\PY{n+nb}{dict}\PY{p}{,} \PY{n}{fun}\PY{p}{)}\PY{p}{)}
\end{Verbatim}
\end{tcolorbox}

    \begin{Verbatim}[commandchars=\\\{\}]
(3.50656581341894, 1.8267621151828408)
    \end{Verbatim}

    a = l1, b = linear separations (brigth lines)

c is the starting angle, d is the grating angle

e is \(\alpha\) and will be a function of the c and d.~

f is \(\beta\) and will be a function of e, a, and b.

g is \(\lambda\) will be a function of e and f.

    \begin{tcolorbox}[breakable, size=fbox, boxrule=1pt, pad at break*=1mm,colback=cellbackground, colframe=cellborder]
\prompt{In}{incolor}{206}{\boxspacing}
\begin{Verbatim}[commandchars=\\\{\}]
\PY{n}{arctan} \PY{o}{=} \PY{n}{atan}
\PY{n}{angle\PYZus{}when\PYZus{}blazed}\PY{p}{,} \PY{n}{angle\PYZus{}when\PYZus{}zeroed} \PY{o}{=} \PY{n}{symbols}\PY{p}{(}\PY{l+s+s2}{\PYZdq{}}\PY{l+s+s2}{a b}\PY{l+s+s2}{\PYZdq{}}\PY{p}{)}
\PY{n}{d\PYZus{}n}\PY{p}{,} \PY{n}{grating\PYZus{}to\PYZus{}red\PYZus{}line} \PY{o}{=} \PY{n}{symbols}\PY{p}{(}\PY{l+s+s2}{\PYZdq{}}\PY{l+s+s2}{c d}\PY{l+s+s2}{\PYZdq{}}\PY{p}{)}
\PY{n}{alpha} \PY{o}{=} \PY{n}{angle\PYZus{}when\PYZus{}blazed} \PY{o}{\PYZhy{}} \PY{n}{angle\PYZus{}when\PYZus{}zeroed}
\PY{n}{beta} \PY{o}{=} \PY{n}{arctan}\PY{p}{(}\PY{n}{d\PYZus{}n} \PY{o}{/} \PY{n}{grating\PYZus{}to\PYZus{}red\PYZus{}line}\PY{p}{)} \PY{o}{\PYZhy{}} \PY{n}{alpha}
\PY{n}{wavelength} \PY{o}{=} \PY{l+m+mi}{10}\PY{o}{*}\PY{o}{*}\PY{l+m+mi}{6} \PY{o}{*} \PY{p}{(}\PY{l+m+mi}{1}\PY{o}{/}\PY{l+m+mi}{1200}\PY{p}{)} \PY{o}{*} \PY{p}{(}\PY{n}{sin}\PY{p}{(}\PY{n}{alpha}\PY{p}{)} \PY{o}{+} \PY{n}{sin}\PY{p}{(}\PY{n}{beta}\PY{p}{)}\PY{p}{)}
\PY{n}{md}\PY{p}{(}\PY{l+s+s2}{\PYZdq{}}\PY{l+s+s2}{Where atan is arctan, a is the angle on rotional stage, b}\PY{l+s+s2}{\PYZdq{}} 
   \PY{l+s+s2}{\PYZdq{}}\PY{l+s+s2}{ is the angle on the rotational stage when normal to the }\PY{l+s+s2}{\PYZdq{}} 
   \PY{l+s+s2}{\PYZdq{}}\PY{l+s+s2}{incoming beam, d is the distance from the diffraction grating}\PY{l+s+s2}{\PYZdq{}}
   \PY{l+s+s2}{\PYZdq{}}\PY{l+s+s2}{ to an arbitrary line normal to the incoming beam,}\PY{l+s+s2}{\PYZdq{}} \PY{o}{+}  
   \PY{l+s+s2}{\PYZdq{}}\PY{l+s+s2}{c is the distance from the spectral feature on the screen}\PY{l+s+s2}{\PYZdq{}} \PY{o}{+} 
    \PY{l+s+s2}{\PYZdq{}}\PY{l+s+s2}{from the incoming beam along the same arbitrary line}\PY{l+s+s2}{\PYZdq{}} \PY{o}{+} 
   \PY{l+s+s2}{\PYZdq{}}\PY{l+s+s2}{ we have for the wavelength in nanometers}\PY{l+s+se}{\PYZbs{}\PYZbs{}}\PY{l+s+s2}{begin}\PY{l+s+si}{\PYZob{}equation\PYZcb{}}\PY{l+s+s2}{\PYZdq{}} \PY{o}{+} 
   \PY{l+s+s2}{\PYZdq{}}\PY{l+s+se}{\PYZbs{}\PYZbs{}}\PY{l+s+s2}{lambda=}\PY{l+s+s2}{\PYZdq{}} \PY{o}{+} \PY{n}{latp}\PY{p}{(}\PY{n}{simplify}\PY{p}{(}\PY{n}{wavelength}\PY{p}{)}\PY{p}{)} \PY{o}{+}\PY{l+s+s2}{\PYZdq{}}\PY{l+s+se}{\PYZbs{}\PYZbs{}}\PY{l+s+s2}{end}\PY{l+s+si}{\PYZob{}equation\PYZcb{}}\PY{l+s+s2}{\PYZdq{}}\PY{p}{)}
\end{Verbatim}
\end{tcolorbox}
 
            
\prompt{Out}{outcolor}{206}{}
    
    Where atan is arctan, a is the angle on rotional stage, b is the angle
on the rotational stage when normal to the incoming beam, d is the
distance from the diffraction grating to an arbitrary line normal to the
incoming beam,c is the distance from the spectral feature on the
screenfrom the incoming beam along the same arbitrary line we have for
the wavelength in
nanometers\begin{equation}\lambda=833.333333333333 \sin{\left(a - b \right)} + 833.333333333333 \sin{\left(- a + b + \operatorname{atan}{\left(\frac{c}{d} \right)} \right)}\end{equation}

    

    \begin{tcolorbox}[breakable, size=fbox, boxrule=1pt, pad at break*=1mm,colback=cellbackground, colframe=cellborder]
\prompt{In}{incolor}{234}{\boxspacing}
\begin{Verbatim}[commandchars=\\\{\}]
\PY{n}{averaged\PYZus{}dict} \PY{o}{=} \PY{p}{(}\PY{n}{average\PYZus{}angle\PYZus{}zeroed} \PY{o}{|} \PY{n}{average\PYZus{}angle\PYZus{}blazed} \PY{o}{|}
            \PY{n}{brighter\PYZus{}further} \PY{o}{|} \PY{p}{\PYZob{}}\PY{l+m+mf}{97.5}\PY{p}{:}\PY{l+m+mf}{.7}\PY{p}{\PYZcb{}} \PY{p}{)}
\PY{n}{lam\PYZus{}brighter\PYZus{}further} \PY{o}{=} \PY{n}{general}\PY{p}{(}\PY{n}{averaged\PYZus{}dict}\PY{p}{,} \PY{n}{wavelength}\PY{p}{)}
\PY{n}{md}\PY{p}{(}\PY{l+s+s2}{\PYZdq{}}\PY{l+s+s2}{We have for the wavelength of the brighter emission line}\PY{l+s+s2}{\PYZdq{}} \PY{o}{+} 
   \PY{l+s+s2}{\PYZdq{}}\PY{l+s+s2}{in the doublet in nm}\PY{l+s+s2}{\PYZdq{}} \PY{o}{+}  \PY{l+s+s2}{\PYZdq{}}\PY{l+s+se}{\PYZbs{}\PYZbs{}}\PY{l+s+s2}{begin}\PY{l+s+si}{\PYZob{}equation\PYZcb{}}\PY{l+s+se}{\PYZbs{}\PYZbs{}}\PY{l+s+s2}{Large}\PY{l+s+se}{\PYZbs{}\PYZbs{}}\PY{l+s+s2}{lambda=}\PY{l+s+s2}{\PYZdq{}} \PY{o}{+} 
   \PY{n}{latp}\PY{p}{(}\PY{n}{lam\PYZus{}brighter\PYZus{}further}\PY{p}{[}\PY{l+m+mi}{0}\PY{p}{]}\PY{p}{)} \PY{o}{+} \PY{l+s+s2}{\PYZdq{}}\PY{l+s+se}{\PYZbs{}\PYZbs{}}\PY{l+s+s2}{pm}\PY{l+s+s2}{\PYZdq{}} \PY{o}{+} 
   \PY{n}{latp}\PY{p}{(}\PY{n}{lam\PYZus{}brighter\PYZus{}further}\PY{p}{[}\PY{l+m+mi}{1}\PY{p}{]}\PY{p}{)} \PY{o}{+} \PY{l+s+s2}{\PYZdq{}}\PY{l+s+se}{\PYZbs{}\PYZbs{}}\PY{l+s+s2}{end}\PY{l+s+si}{\PYZob{}equation\PYZcb{}}\PY{l+s+s2}{\PYZdq{}}\PY{p}{)}
\end{Verbatim}
\end{tcolorbox}
 
            
\prompt{Out}{outcolor}{234}{}
    
    We have for the wavelength of the brighter emission linein the doublet
in
nm\begin{equation}\Large\lambda=508.252648723968\pm2.74364673383703\end{equation}

    

    \begin{tcolorbox}[breakable, size=fbox, boxrule=1pt, pad at break*=1mm,colback=cellbackground, colframe=cellborder]
\prompt{In}{incolor}{224}{\boxspacing}
\begin{Verbatim}[commandchars=\\\{\}]
\PY{n}{averaged\PYZus{}dict} \PY{o}{=} \PY{p}{(}\PY{n}{average\PYZus{}angle\PYZus{}zeroed} \PY{o}{|} \PY{n}{average\PYZus{}angle\PYZus{}blazed} \PY{o}{|}
            \PY{n}{dimmer\PYZus{}closer} \PY{o}{|} \PY{p}{\PYZob{}}\PY{l+m+mf}{97.5}\PY{p}{:}\PY{l+m+mf}{.7}\PY{p}{\PYZcb{}} \PY{p}{)}
\PY{n}{lam\PYZus{}dimmer\PYZus{}closer} \PY{o}{=} \PY{n}{general}\PY{p}{(}\PY{n}{averaged\PYZus{}dict}\PY{p}{,} \PY{n}{wavelength}\PY{p}{)}
\PY{n}{md}\PY{p}{(}\PY{l+s+s2}{\PYZdq{}}\PY{l+s+s2}{We have for the wavelength of the brighter emission line in the}\PY{l+s+s2}{\PYZdq{}} \PY{o}{+}
   \PY{l+s+s2}{\PYZdq{}}\PY{l+s+s2}{ doublet in nm}\PY{l+s+se}{\PYZbs{}\PYZbs{}}\PY{l+s+s2}{begin}\PY{l+s+si}{\PYZob{}equation\PYZcb{}}\PY{l+s+se}{\PYZbs{}\PYZbs{}}\PY{l+s+s2}{Large}\PY{l+s+se}{\PYZbs{}\PYZbs{}}\PY{l+s+s2}{lambda=}\PY{l+s+s2}{\PYZdq{}} \PY{o}{+} 
   \PY{n}{latp}\PY{p}{(}\PY{n}{lam\PYZus{}dimmer\PYZus{}closer}\PY{p}{[}\PY{l+m+mi}{0}\PY{p}{]}\PY{p}{)} \PY{o}{+} 
   \PY{l+s+s2}{\PYZdq{}}\PY{l+s+se}{\PYZbs{}\PYZbs{}}\PY{l+s+s2}{pm}\PY{l+s+s2}{\PYZdq{}} \PY{o}{+} \PY{n}{latp}\PY{p}{(}\PY{n}{lam\PYZus{}dimmer\PYZus{}closer}\PY{p}{[}\PY{l+m+mi}{1}\PY{p}{]}\PY{p}{)} \PY{o}{+} \PY{l+s+s2}{\PYZdq{}}\PY{l+s+se}{\PYZbs{}\PYZbs{}}\PY{l+s+s2}{end}\PY{l+s+si}{\PYZob{}equation\PYZcb{}}\PY{l+s+s2}{\PYZdq{}}\PY{p}{)}
\end{Verbatim}
\end{tcolorbox}
 
            
\prompt{Out}{outcolor}{224}{}
    
    We have for the wavelength of the brighter emission line in the doublet
in
nm\begin{equation}\Large\lambda=474.989240424089\pm2.88615877888694\end{equation}

    

    \begin{tcolorbox}[breakable, size=fbox, boxrule=1pt, pad at break*=1mm,colback=cellbackground, colframe=cellborder]
\prompt{In}{incolor}{237}{\boxspacing}
\begin{Verbatim}[commandchars=\\\{\}]
\PY{n}{e} \PY{o}{=} \PY{n}{d} \PY{o}{\PYZhy{}} \PY{n}{c}  
\PY{n}{f} \PY{o}{=} \PY{n}{arctan}\PY{p}{(} \PY{n}{b} \PY{o}{/} \PY{n}{a}\PY{p}{)} \PY{o}{\PYZhy{}} \PY{n}{e}
\PY{n}{g} \PY{o}{=} \PY{l+m+mi}{10}\PY{o}{*}\PY{o}{*}\PY{o}{\PYZhy{}}\PY{l+m+mi}{3} \PY{o}{/} \PY{p}{(}\PY{l+m+mi}{1200}\PY{p}{)} \PY{o}{*} \PY{p}{(}\PY{n}{sin}\PY{p}{(}\PY{n}{e}\PY{p}{)} \PY{o}{+} \PY{n}{sin}\PY{p}{(}\PY{n}{f}\PY{p}{)}\PY{p}{)}
\PY{n}{b\PYZus{}arr} \PY{o}{=} \PY{p}{[}\PY{l+m+mf}{74.0}\PY{p}{,} \PY{l+m+mf}{75.5}\PY{p}{,} \PY{l+m+mf}{86.0}\PY{p}{]}
\PY{n}{c\PYZus{}arr} \PY{o}{=} \PY{n}{np}\PY{o}{.}\PY{n}{pi} \PY{o}{*} \PY{n}{np}\PY{o}{.}\PY{n}{array}\PY{p}{(}\PY{p}{[}\PY{l+m+mi}{95}\PY{o}{+}\PY{l+m+mi}{35}\PY{o}{/}\PY{l+m+mi}{60}\PY{p}{,} \PY{l+m+mi}{95} \PY{o}{+} \PY{l+m+mi}{25}\PY{o}{/}\PY{l+m+mi}{60}\PY{p}{,} \PY{l+m+mi}{96} \PY{o}{+} \PY{l+m+mi}{5}\PY{o}{/}\PY{l+m+mi}{60}\PY{p}{]}\PY{p}{)} \PY{o}{/} \PY{l+m+mi}{180}
\PY{n}{d\PYZus{}arr} \PY{o}{=} \PY{n}{np}\PY{o}{.}\PY{n}{pi} \PY{o}{*} \PY{n}{np}\PY{o}{.}\PY{n}{array}\PY{p}{(}\PY{p}{[}\PY{l+m+mi}{97}\PY{o}{+}\PY{l+m+mi}{55}\PY{o}{/}\PY{l+m+mi}{60}\PY{p}{,} \PY{l+m+mi}{98} \PY{o}{+} \PY{l+m+mi}{10}\PY{o}{/}\PY{l+m+mi}{60}\PY{p}{,} \PY{l+m+mi}{97} \PY{o}{+} \PY{l+m+mi}{50}\PY{o}{/}\PY{l+m+mi}{60}\PY{p}{]}\PY{p}{)} \PY{o}{/} \PY{l+m+mi}{180}
\PY{n}{angle\PYZus{}unc} \PY{o}{=} \PY{l+m+mi}{5} \PY{o}{*} \PY{n}{np}\PY{o}{.}\PY{n}{pi} \PY{o}{/} \PY{p}{(} \PY{l+m+mi}{60} \PY{o}{*} \PY{l+m+mi}{180} \PY{p}{)}
\PY{n}{output} \PY{o}{=} \PY{p}{[}\PY{n}{general}\PY{p}{(}\PY{p}{\PYZob{}}\PY{l+m+mf}{97.5}\PY{p}{:}\PY{l+m+mf}{.7}\PY{p}{,} \PY{n}{b\PYZus{}arr}\PY{p}{[}\PY{n}{i}\PY{p}{]}\PY{p}{:}\PY{l+m+mf}{.7}\PY{p}{,} \PY{n}{c\PYZus{}arr}\PY{p}{[}\PY{n}{i}\PY{p}{]}\PY{p}{:}\PY{n}{angle\PYZus{}unc}\PY{p}{,} 
                \PY{n}{d\PYZus{}arr}\PY{p}{[}\PY{n}{i}\PY{p}{]}\PY{p}{:}\PY{n}{angle\PYZus{}unc}\PY{p}{\PYZcb{}}\PY{p}{,} \PY{n}{g}\PY{p}{)} \PY{k}{for} \PY{n}{i} \PY{o+ow}{in} \PY{n+nb}{range}\PY{p}{(}\PY{n+nb}{len}\PY{p}{(}\PY{n}{b\PYZus{}arr}\PY{p}{)}\PY{p}{)}\PY{p}{]}
\PY{n}{md}\PY{p}{(}\PY{l+s+s2}{\PYZdq{}}\PY{l+s+s2}{We have for the wavelengths and uncertainties of the doublet}\PY{l+s+s2}{\PYZdq{}}
   \PY{l+s+s2}{\PYZdq{}}\PY{l+s+se}{\PYZbs{}\PYZbs{}}\PY{l+s+s2}{begin}\PY{l+s+si}{\PYZob{}equation\PYZcb{}}\PY{l+s+s2}{\PYZdq{}} \PY{o}{+} \PY{n}{latp}\PY{p}{(}\PY{n}{output}\PY{p}{)} \PY{o}{+} \PY{l+s+s2}{\PYZdq{}}\PY{l+s+se}{\PYZbs{}\PYZbs{}}\PY{l+s+s2}{end}\PY{l+s+si}{\PYZob{}equation\PYZcb{}}\PY{l+s+s2}{\PYZdq{}}\PY{o}{+} 
   \PY{l+s+s2}{\PYZdq{}}\PY{l+s+s2}{and for the width of each slit and their uncertainty in microns, respectively}\PY{l+s+s2}{\PYZdq{}} \PY{o}{+} 
   \PY{l+s+s2}{\PYZdq{}}\PY{l+s+se}{\PYZbs{}\PYZbs{}}\PY{l+s+s2}{begin}\PY{l+s+si}{\PYZob{}equation\PYZcb{}}\PY{l+s+s2}{\PYZdq{}} \PY{o}{+} \PY{n+nb}{str}\PY{p}{(}\PY{n}{slit\PYZus{}widths}\PY{p}{)} \PY{o}{+} \PY{l+s+s2}{\PYZdq{}}\PY{l+s+se}{\PYZbs{}\PYZbs{}}\PY{l+s+s2}{end}\PY{l+s+si}{\PYZob{}equation\PYZcb{}}\PY{l+s+s2}{\PYZdq{}}\PY{p}{)}
\end{Verbatim}
\end{tcolorbox}
 
            
\prompt{Out}{outcolor}{237}{}
    
    We have for the wavelengths and uncertainties of the
doublet\begin{equation}\left[ \left( 5.1028905792447 \cdot 10^{-7}, \  3.92228677149986 \cdot 10^{-9}\right), \  \left( 5.17994675481773 \cdot 10^{-7}, \  3.88705235226923 \cdot 10^{-9}\right), \  \left( 5.57351711971439 \cdot 10^{-7}, \  3.47638104232865 \cdot 10^{-9}\right)\right]\end{equation}and
for the width of each slit and their uncertainty in microns,
respectively\begin{equation}{15.0: 0.5, 19.5: 0.5, 8.0: 0.5}\end{equation}

    

    \hypertarget{discussion}{%
\section{Discussion}\label{discussion}}

As expected, the spectral resolution somewhat increased with smaller
slit widths, but the doublet reflected by the diffraction grating onto
the screen was noticeably dimmer. Although a smaller slit width can
allow for a higher spectral resolution via the grating equation, it can
also act as a field stop and decrease the flux of light onto the screen.
For instance, if one were using a CCD instead of a screen, then for a
dim light source then the total photon counts would decrease,
potentially impacting the science goals of the experiment or
observation. Thus, a smaller slit width is not always monotonically
better, and one must carefully weigh the trade-off between photon
throughput and spectral resolution. For instance, a slit is too thin may
create unnecessary uncertainty in the depth of a spectral feature that
is of interest to the investigation.

The focal lengths of collimating biconvex lenses were not particularly
significant because as long as an image was formed on the screen, then
changing the distance of the screen would simply make the image larger,
smaller, or out of focus. Any damage or aberrations in any of the
components in the optical train, or misallignment in the vertical plane
would be detrimental to the spectral resolution. In practice, systematic
errors and noise due to the process with which we measured each distance
used in the wavelength calculation were easily the largest sources of
error.

Similar findings would be expected for the zeroeth order, but of course
without spectral resolution playing a role since the zeroeth order
cannot resolve spectral features.

Were part 2 of the experiment possible without time constraints, the
values measured would be substituted into the Equation for spectral
resolution derived above in this report, and it would be expected to be
smaller than the same value of diffraction grating spectrographs.

    \begin{tcolorbox}[breakable, size=fbox, boxrule=1pt, pad at break*=1mm,colback=cellbackground, colframe=cellborder]
\prompt{In}{incolor}{225}{\boxspacing}
\begin{Verbatim}[commandchars=\\\{\}]
\PY{n}{md}\PY{p}{(}\PY{l+s+s2}{\PYZdq{}}\PY{l+s+s2}{If we instead seek to find a singular wavelength for the entire}\PY{l+s+s2}{\PYZdq{}}
  \PY{o}{+} \PY{l+s+s2}{\PYZdq{}}\PY{l+s+s2}{ doublet then we have in meters}\PY{l+s+se}{\PYZbs{}\PYZbs{}}\PY{l+s+s2}{begin}\PY{l+s+si}{\PYZob{}equation\PYZcb{}}\PY{l+s+se}{\PYZbs{}\PYZbs{}}\PY{l+s+s2}{large}\PY{l+s+se}{\PYZbs{}\PYZbs{}}\PY{l+s+s2}{lambda=}\PY{l+s+s2}{\PYZdq{}} \PY{o}{+} 
  \PY{n}{latp}\PY{p}{(}\PY{n}{np}\PY{o}{.}\PY{n}{average}\PY{p}{(}\PY{p}{[}\PY{n}{x}\PY{p}{[}\PY{l+m+mi}{0}\PY{p}{]} \PY{k}{for} \PY{n}{x} \PY{o+ow}{in} \PY{n}{output}\PY{p}{]}\PY{p}{)}\PY{p}{)} \PY{o}{+} \PY{l+s+s2}{\PYZdq{}}\PY{l+s+se}{\PYZbs{}\PYZbs{}}\PY{l+s+s2}{pm}\PY{l+s+s2}{\PYZdq{}} \PY{o}{+} 
   \PY{n}{latp}\PY{p}{(}\PY{n}{np}\PY{o}{.}\PY{n}{average}\PY{p}{(}\PY{p}{[}\PY{n}{x}\PY{p}{[}\PY{l+m+mi}{1}\PY{p}{]} \PY{k}{for} \PY{n}{x} \PY{o+ow}{in} \PY{n}{output}\PY{p}{]}\PY{o}{/}\PY{n}{np}\PY{o}{.}\PY{n}{sqrt}\PY{p}{(}\PY{n+nb}{len}\PY{p}{(}\PY{n}{output}\PY{p}{)}\PY{p}{)}\PY{p}{)}\PY{p}{)} \PY{o}{+} 
  \PY{l+s+s2}{\PYZdq{}}\PY{l+s+se}{\PYZbs{}\PYZbs{}}\PY{l+s+s2}{end}\PY{l+s+si}{\PYZob{}equation\PYZcb{}}\PY{l+s+s2}{\PYZdq{}}\PY{p}{)} 
\end{Verbatim}
\end{tcolorbox}
 
            
\prompt{Out}{outcolor}{225}{}
    
    If we instead seek to find a singular wavelength for the entire doublet
then we have in
nm\begin{equation}\large\lambda=5.28545148459227 \cdot 10^{-7}\pm2.17193785863177 \cdot 10^{-9}\end{equation}

    

    Each member of the lab group contributed equally, since we implemented a
system where after each set of measurements, we would rotate who was
participating in what role.

    \begin{tcolorbox}[breakable, size=fbox, boxrule=1pt, pad at break*=1mm,colback=cellbackground, colframe=cellborder]
\prompt{In}{incolor}{239}{\boxspacing}
\begin{Verbatim}[commandchars=\\\{\}]
\PY{o}{!}jupyter nbconvert \PYZhy{}\PYZhy{}to latex Lab4.ipynb
\end{Verbatim}
\end{tcolorbox}

    \begin{Verbatim}[commandchars=\\\{\}]
[NbConvertApp] Converting notebook Lab4.ipynb to latex
[NbConvertApp] Writing 107514 bytes to Lab4.tex
    \end{Verbatim}

    \begin{tcolorbox}[breakable, size=fbox, boxrule=1pt, pad at break*=1mm,colback=cellbackground, colframe=cellborder]
\prompt{In}{incolor}{ }{\boxspacing}
\begin{Verbatim}[commandchars=\\\{\}]

\end{Verbatim}
\end{tcolorbox}

    \begin{tcolorbox}[breakable, size=fbox, boxrule=1pt, pad at break*=1mm,colback=cellbackground, colframe=cellborder]
\prompt{In}{incolor}{ }{\boxspacing}
\begin{Verbatim}[commandchars=\\\{\}]

\end{Verbatim}
\end{tcolorbox}

    \begin{tcolorbox}[breakable, size=fbox, boxrule=1pt, pad at break*=1mm,colback=cellbackground, colframe=cellborder]
\prompt{In}{incolor}{ }{\boxspacing}
\begin{Verbatim}[commandchars=\\\{\}]

\end{Verbatim}
\end{tcolorbox}

    \begin{tcolorbox}[breakable, size=fbox, boxrule=1pt, pad at break*=1mm,colback=cellbackground, colframe=cellborder]
\prompt{In}{incolor}{ }{\boxspacing}
\begin{Verbatim}[commandchars=\\\{\}]

\end{Verbatim}
\end{tcolorbox}

    \begin{tcolorbox}[breakable, size=fbox, boxrule=1pt, pad at break*=1mm,colback=cellbackground, colframe=cellborder]
\prompt{In}{incolor}{ }{\boxspacing}
\begin{Verbatim}[commandchars=\\\{\}]

\end{Verbatim}
\end{tcolorbox}

    \begin{tcolorbox}[breakable, size=fbox, boxrule=1pt, pad at break*=1mm,colback=cellbackground, colframe=cellborder]
\prompt{In}{incolor}{ }{\boxspacing}
\begin{Verbatim}[commandchars=\\\{\}]

\end{Verbatim}
\end{tcolorbox}

    \begin{tcolorbox}[breakable, size=fbox, boxrule=1pt, pad at break*=1mm,colback=cellbackground, colframe=cellborder]
\prompt{In}{incolor}{ }{\boxspacing}
\begin{Verbatim}[commandchars=\\\{\}]

\end{Verbatim}
\end{tcolorbox}

    \begin{tcolorbox}[breakable, size=fbox, boxrule=1pt, pad at break*=1mm,colback=cellbackground, colframe=cellborder]
\prompt{In}{incolor}{ }{\boxspacing}
\begin{Verbatim}[commandchars=\\\{\}]

\end{Verbatim}
\end{tcolorbox}

    \begin{tcolorbox}[breakable, size=fbox, boxrule=1pt, pad at break*=1mm,colback=cellbackground, colframe=cellborder]
\prompt{In}{incolor}{ }{\boxspacing}
\begin{Verbatim}[commandchars=\\\{\}]

\end{Verbatim}
\end{tcolorbox}

    


    % Add a bibliography block to the postdoc
    
    
    
\end{document}
